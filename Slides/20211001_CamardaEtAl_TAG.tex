\documentclass[table,xcdraw,dvipsnames]{beamer}
% \documentclass[mathserif]{beamer}
\usepackage{tikz}
\usetikzlibrary{arrows.meta}
\definecolor{my-grey}{RGB}{220,220,220}
\definecolor{myDarkRed}{RGB}{153,0,0}
\definecolor{myLightRed}{RGB}{255,102,102}
\definecolor{myDarkBlue}{RGB}{0,0,255}
\definecolor{myLightBlue}{RGB}{102,178,255}
\usepackage{booktabs}


\usepackage{soul}
\usetheme{Frankfurt}
\usecolortheme{beaver}
\usepackage{abraces}
\usepackage{marvosym}
\usepackage{multirow}
\usepackage{textpos}
\usepackage{amsmath, amssymb, bm}
%\usepackage{kbordermatrix}
\usepackage{tikz}
\usetikzlibrary{positioning}
%% \usepackage{euler}
\usepackage{animate}
\newcommand{\gooditem}[1]{\setbeamercolor{item}{fg=Green}\item #1} 
\newcommand{\pooritem}[1]{\setbeamercolor{item}{fg=Red}\item #1} 
\usepackage[normalem]{ulem}
% \usepackage[table,xcdraw,dvipsnames]{xcolor}

\newcommand*\rot[1]{\rotatebox{90}{#1}}

%\everymath{\displaystyle}
%\everymath{\color{Blue}}


\makeatletter
\setbeamertemplate{footline}
{
  \leavevmode%
  \hbox{%
  \begin{beamercolorbox}[wd=.50\paperwidth,ht=2.25ex,dp=1ex,center]{author in head/foot}%
    \usebeamerfont{author in head/foot}\insertshortauthor%% \beamer@ifempty{\insertshortinstitute}{}{(\insertshortinstitute)}
  \end{beamercolorbox}%
  \hskip2pt%
  \begin{beamercolorbox}[wd=.50\paperwidth,ht=2.25ex,dp=1ex,center]{title in head/foot}%
    \usebeamerfont{title in head/foot}\insertshorttitle~~~~~~~~~~~~~~~~\insertframenumber
  \end{beamercolorbox}%
  %% \begin{beamercolorbox}[wd=.20\paperwidth,ht=2.25ex,dp=1ex,right]{date in head/foot}%
  %%   \usebeamerfont{date in head/foot}\insertshortdate{}\hspace*{2em}
  %%   \insertframenumber\hspace*{2ex} 
  %% \end{beamercolorbox}
}%
  \vskip0pt%
}
\makeatother



\def\checkmark{\tikz\fill[scale=0.4](0,.35) -- (.25,0) -- (1,.7) -- (.25,.15) -- cycle;} 

\definecolor{lightblue}{rgb}{0.145,0.6666,1} % Defines the color used for content box headers
\definecolor{Red}{rgb}{0.9,0.15,0}
\definecolor{Blue}{rgb}{0.1,0.3,1}
% \definecolor{Blue}{rgb}{0.1,0.7,1}
\definecolor{Green}{rgb}{0.3,0.8,0.15}
\definecolor{Lightgray}{rgb}{0.86,0.86,0.86}
\setbeamercolor{block title}{bg=Red!30,fg=black}


\title[Comparing approaches in modelling 2020 mortality]
{\LARGE{Comparing approaches in modelling 2020 \\overall mortality by sex and age}} 


\author[Camarda et al.$\qquad\quad\qquad\quad\qquad\quad\qquad\quad\qquad\quad$October 2021]
{\large{\sl{Carlo Giovanni Camarda}}\\
\large{\sl{Tim Riffe}}\\
\large{\sl{Simona Bignani}}\\
\large{\sl{Enrique Acosta}}
}


\date[Oct 2021]{
$\,$\\$\,$\\ \textsl{TAG Working Group I}\\
$\,$\\
\vspace{0.1cm}
\footnotesize{October 1st, 2021}} % (optional) 

\beamertemplatenavigationsymbolsempty
\begin{document}

\begin{frame}[plain]
\titlepage
\end{frame}

\addtobeamertemplate{frametitle}{}{%
\begin{textblock*}{100mm}(0.999\textwidth,-1.4cm)
\includegraphics[scale=0.07]{ined_Hs_couleur}
\end{textblock*}}

% \section*{}
% \begin{frame}{Outline}
%   \tableofcontents
% \end{frame}

% % table of contents only at the beginning
% \section[Outline]{}
% \frame{
% \frametitle{Outline}
% \tableofcontents
% }

% table of contents popping up at
% the beginning of each subsection:
\AtBeginSection[] {
  \begin{frame}<beamer>{Outline}
    % \begin{small}
    \tableofcontents[currentsection]
    % \end{small}
  \end{frame}
}




\begin{frame}[fragile]\frametitle{Data and aim}
	\vspace{-0.2cm}
\begin{itemize}
	\item<1-> Death and exposures by 
	\begin{itemize}
		\item five years age-group: 0-4, 5-9, $\ldots$, 85+
		\item females and males separately 
	\end{itemize}	
	\item<2->  2019 (baseline):
	\begin{itemize}
		\item \textbf{Source:} Global Health Estimates (GHE)
		\item \textbf{Availability:} all 194 populations
	\end{itemize}
	\item<3->  2020:
	\begin{itemize}
		\item \textbf{Source:} Various
		\item \textbf{Availability:} only 60 populations
	\end{itemize}
	\item<4->  Information provided by William et al.
	\begin{itemize}
		\item Overall excess deaths in 2020 for all populations 
		\item 8 clusters for all populations, based on external information
	\end{itemize}
\end{itemize}
\onslide<5>{
\begin{alertblock}{\centering \textbf{AIM}}
Estimate 2020 mortality by age and sex for pop.~with no data
\end{alertblock}	
}
\end{frame}


\begin{frame}[fragile]\frametitle{The two approaches}
	\vspace{-0.2cm}
	
	\begin{itemize}
		\item<1-> William's approach:
		\begin{enumerate}
			\item empirical ratio between mortality in 2019 and 2020 for population with both information
			\item group of these ratios based on mentioned clusters
			\item creation of a smooth cluster-specific distribution of ratios
			\item extract random ratio and apply it to 2019 mortality for population with no information in 2020
		\end{enumerate}
	\item<2-> Spin-off (our) approach:
		\begin{enumerate}
		\item for a given population with both information within a cluster $k$:
	$$
	    \eta^{2020}(x) = \eta^{2019}(x) + c + \delta^{k}(x)
	$$
	with $\eta(x)$ and $\delta^{k}(x)$ assumed to be smooth and $\sum \delta^{k}(x) = 0$
		\item apply cluster-specific age-factor $\delta^{k}(x)$ to population with no information in 2020
		\item (uncertainty still to be included)
	\end{enumerate}
\item<3-> Final common step: redistribution of estimated 2020 deaths to match ``known'' overall excess mortality deaths
	 \end{itemize}	
 

\end{frame}


\begin{frame}[fragile]\frametitle{Clusters}
	\only<1>{
		\begin{center}
			\hspace*{-2cm}\includegraphics[scale=.6]{Figures/wm_clusters.pdf}
		\end{center}
	}
	\only<2>{
		\begin{scriptsize}
			\begin{table}[]
				\begin{tabular}{clrrr}
					Cluster                                            & \multicolumn{1}{c}{pop w/ data}                                                                           & \multicolumn{1}{l}{\begin{tabular}[c]{@{}l@{}}\# pop \\ w/ data\end{tabular}} & \multicolumn{1}{c}{\begin{tabular}[c]{@{}c@{}}\# pop \\ w/o data\end{tabular}} & \multicolumn{1}{l}{\begin{tabular}[c]{@{}c@{}}\% of pop. \\ w/ data\end{tabular}} \\ \hline
					\rowcolor[HTML]{C0C0C0} 
					1                                                  & \begin{tabular}[c]{@{}l@{}}CYP,DNK,EST,FIN,ISL,\\ JPN,KOR,LUX,NOR\end{tabular}                            & 9                                                                             & 4                                                                              & 69\%                                                              \\
					2                                                  & \begin{tabular}[c]{@{}l@{}}ALB,AND,BEL,BGR,CZE,\\ ESP,ITA,LTU,MDA,POL,ROU,\\ RUS,SRB,SVN,USA\end{tabular} & 15                                                                            & 4                                                                              & 79\%                                                              \\
					\rowcolor[HTML]{C0C0C0} 
					3                                                  & BRA,IRQ,ZAF,COL                                                                                           & 4                                                                             & 65                                                                             & $\,$6\%                                                             \\
					4                                                  & \begin{tabular}[c]{@{}l@{}}AUT,CHE,CHL,FRA,GBR,\\ GEO,HRV,HUN,MNE,NLD,\\ PRT,SVK,SWE,UKR\end{tabular}     & 14                                                                            & 0                                                                              & 100\%                                                              \\
					\rowcolor[HTML]{C0C0C0} 
					5                                                  & CAN,DEU,GRC,LVA,MLT                                                                                       & 5                                                                             & 1                                                                              & 83\%                                                              \\
					6                                                  & CRI,ISR,PRY,TUN                                                                                           & 4                                                                             & 51                                                                             & 7\%                                                             \\
					\rowcolor[HTML]{C0C0C0} 
					7                                                  & AUS,IRL,MUS,NZL,URY                                                                                       & 5                                                                             & 7                                                                              & 42\%                                                              \\
					8                                                  & ARM,ECU,MEX,PER                                                                                           & 4                                                                             & 2                                                                              & 67\%                                                              \\ \hline
					\rowcolor[HTML]{C0C0C0} 
					\multicolumn{1}{l}{\cellcolor[HTML]{C0C0C0}\textbf{Totals}} &                                                                                                           & 60                                                                            & 134                                                                            & 31\%                                                             
				\end{tabular}
			\end{table}
\end{scriptsize}
}
\end{frame}





\begin{frame}[fragile]\frametitle{Cluster-specific $\delta(x)$}
	\begin{center}
		%\hspace*{-2cm}
		\includegraphics[scale=.35]{Figures/ClusterDeltas.pdf}
	\end{center}
\end{frame}


\begin{frame}[fragile]\frametitle{Log-mortality in 2019: in-sample fit}
	\vspace*{-0.2cm}
	\begin{center}
		%\hspace*{-2cm}
		\includegraphics[scale=.3]{Figures/ActualFittedLogMortality2019.pdf}
	\end{center}
\end{frame}

\begin{frame}[fragile]\frametitle{Log-mortality in 2020: in-sample fit}
	\vspace*{-0.2cm}
	\begin{center}
		%\hspace*{-2cm}
		\includegraphics[scale=.3]{Figures/ActualFittedLogMortality2020.pdf}
	\end{center}
\end{frame}


\begin{frame}[fragile]\frametitle{Out-of-sample comparison, Cluster 1 (69\%)}
	\vspace*{-0.2cm}
	\begin{center}
		%		\includegraphics[scale=.3, page=44]{Figures/in_sample_compare2.pdf}
		%		\includegraphics[scale=.3, page=33]{Figures/in_sample_compare2.pdf}
		\includegraphics[scale=.18, page=44]{Figures/in_sample_compare2.pdf}
		\includegraphics[scale=.18, page=44]{Figures/all_diffs.pdf}
		\includegraphics[scale=.18, page=44]{Figures/all_sex_ratios.pdf}

		
		\includegraphics[scale=.18, page=33]{Figures/in_sample_compare2.pdf}
		\includegraphics[scale=.18, page=33]{Figures/all_diffs.pdf}
		\includegraphics[scale=.18, page=33]{Figures/all_sex_ratios.pdf}
	\end{center}
\end{frame}

\begin{frame}[fragile]\frametitle{Out-of-sample comparison, Cluster 2 (79\%)}
	\vspace*{-0.2cm}
	\begin{center}
%		\includegraphics[scale=.3, page=55]{Figures/in_sample_compare2.pdf}
%		\includegraphics[scale=.3, page=21]{Figures/in_sample_compare2.pdf}
		\includegraphics[scale=.18, page=55]{Figures/in_sample_compare2.pdf}
		\includegraphics[scale=.18, page=55]{Figures/all_diffs.pdf}
		\includegraphics[scale=.18, page=55]{Figures/all_sex_ratios.pdf}
		
		
		\includegraphics[scale=.18, page=21]{Figures/in_sample_compare2.pdf}
		\includegraphics[scale=.18, page=21]{Figures/all_diffs.pdf}
		\includegraphics[scale=.18, page=21]{Figures/all_sex_ratios.pdf}
	\end{center}
\end{frame}

\begin{frame}[fragile]\frametitle{Out-of-sample comparison, Cluster 3 (6\%)}
	\vspace*{-0.2cm}
	\begin{center}
%		\includegraphics[scale=.3, page=24]{Figures/in_sample_compare2.pdf}
%		\includegraphics[scale=.3, page=53]{Figures/in_sample_compare2.pdf}
		\includegraphics[scale=.18, page=24]{Figures/in_sample_compare2.pdf}
		\includegraphics[scale=.18, page=24]{Figures/all_diffs.pdf}
		\includegraphics[scale=.18, page=24]{Figures/all_sex_ratios.pdf}
		
		
		\includegraphics[scale=.18, page=53]{Figures/in_sample_compare2.pdf}
		\includegraphics[scale=.18, page=53]{Figures/all_diffs.pdf}
		\includegraphics[scale=.18, page=53]{Figures/all_sex_ratios.pdf}
	\end{center}
\end{frame}

\begin{frame}[fragile]\frametitle{Out-of-sample comparison, Cluster 4 (100\%)}
	\vspace*{-0.2cm}
	\begin{center}
%		\includegraphics[scale=.3, page=60]{Figures/in_sample_compare2.pdf}
%		\includegraphics[scale=.2]{Figures/Empty.pdf}
		\includegraphics[scale=.18, page=60]{Figures/in_sample_compare2.pdf}
		\includegraphics[scale=.18, page=60]{Figures/all_diffs.pdf}
		\includegraphics[scale=.18, page=60]{Figures/all_sex_ratios.pdf}
		
		
		\includegraphics[scale=.1]{Figures/Empty.pdf}
		\includegraphics[scale=.1]{Figures/Empty.pdf}
		\includegraphics[scale=.1]{Figures/Empty.pdf}
	\end{center}
\end{frame}

\begin{frame}[fragile]\frametitle{Out-of-sample comparison, Cluster 5 (83\%)}
	\vspace*{-0.2cm}
	\begin{center}
%		\includegraphics[scale=.3, page=30]{Figures/in_sample_compare2.pdf}
%		\includegraphics[scale=.3, page=108]{Figures/in_sample_compare2.pdf}
		\includegraphics[scale=.18, page=30]{Figures/in_sample_compare2.pdf}
		\includegraphics[scale=.18, page=30]{Figures/all_diffs.pdf}
		\includegraphics[scale=.18, page=30]{Figures/all_sex_ratios.pdf}
		
		
		\includegraphics[scale=.18, page=108]{Figures/in_sample_compare2.pdf}
		\includegraphics[scale=.18, page=108]{Figures/all_diffs.pdf}
		\includegraphics[scale=.18, page=108]{Figures/all_sex_ratios.pdf}
	\end{center}
\end{frame}


\begin{frame}[fragile]\frametitle{Out-of-sample comparison, Cluster 6 (7\%)}
	\vspace*{-0.2cm}
	\begin{center}
%		\includegraphics[scale=.3, page=84]{Figures/in_sample_compare2.pdf}
%		\includegraphics[scale=.3, page=79]{Figures/in_sample_compare2.pdf}
		\includegraphics[scale=.18, page=84]{Figures/in_sample_compare2.pdf}
		\includegraphics[scale=.18, page=84]{Figures/all_diffs.pdf}
		\includegraphics[scale=.18, page=84]{Figures/all_sex_ratios.pdf}
		
		
		\includegraphics[scale=.18, page=79]{Figures/in_sample_compare2.pdf}
		\includegraphics[scale=.18, page=79]{Figures/all_diffs.pdf}
		\includegraphics[scale=.18, page=79]{Figures/all_sex_ratios.pdf}
	\end{center}
\end{frame}


\begin{frame}[fragile]\frametitle{Out-of-sample comparison, Cluster 7 (42\%)}
	\vspace*{-0.2cm}
	\begin{center}
%		\includegraphics[scale=.3, page=80]{Figures/in_sample_compare2.pdf}
%		\includegraphics[scale=.3, page=140]{Figures/in_sample_compare2.pdf}
		\includegraphics[scale=.18, page=80]{Figures/in_sample_compare2.pdf}
		\includegraphics[scale=.18, page=80]{Figures/all_diffs.pdf}
		\includegraphics[scale=.18, page=80]{Figures/all_sex_ratios.pdf}
		
		
		\includegraphics[scale=.18, page=140]{Figures/in_sample_compare2.pdf}
		\includegraphics[scale=.18, page=140]{Figures/all_diffs.pdf}
		\includegraphics[scale=.18, page=140]{Figures/all_sex_ratios.pdf}
	\end{center}
\end{frame}


\begin{frame}[fragile]\frametitle{Out-of-sample comparison, Cluster 8 (67\%)}
	\vspace*{-0.2cm}
	\begin{center}
%		\includegraphics[scale=.3, page=112]{Figures/in_sample_compare2.pdf}
%		\includegraphics[scale=.3, page=23]{Figures/in_sample_compare2.pdf}
		\includegraphics[scale=.18, page=112]{Figures/in_sample_compare2.pdf}
		\includegraphics[scale=.18, page=112]{Figures/all_diffs.pdf}
		\includegraphics[scale=.18, page=112]{Figures/all_sex_ratios.pdf}
		
		
		\includegraphics[scale=.18, page=23]{Figures/in_sample_compare2.pdf}
		\includegraphics[scale=.18, page=23]{Figures/all_diffs.pdf}
		\includegraphics[scale=.18, page=23]{Figures/all_sex_ratios.pdf}
	\end{center}
\end{frame}


\begin{frame}[fragile]\frametitle{Concluding remarks}
	\vspace{-0.2cm}
	\begin{itemize}
		\item<1-> \textbf{Adequate coherence between WM and spin-off group outcomes}
		\bigskip
		\item<2-> More data are now available in 2020 (model can be easily accommodated)
		\item<3-> Diverse data sources contains diverse age-grouping structure
		\item<4-> Include uncertainty measures, coming from $\delta^{k}(x)$ \& $\eta(x)$
		\item<5-> Should we aim outcomes by single year of age?
		\item<6-> How sensible the model is with respect to the clusters and/or overall excess deaths?
		\medskip
		\item<7-> \emph{Something else from your side?}
	\end{itemize}
\end{frame}





%\begin{frame}[fragile]\frametitle{A schematic illustration: model \emph{Common}}
%	\vspace{-0.2cm}
%	\only<1>{
%		\begin{center}
%			\includegraphics[scale=.18]{Figures/SimulEta0FSex.pdf}
%			\includegraphics[scale=.18]{Figures/SimulEta0MSex.pdf}
%			\includegraphics[scale=.18]{Figures/SimulDelta0Sex.pdf}
%			\includegraphics[scale=.18]{Figures/Simuls0Sex.pdf}
%		\end{center}
%	}
%	\only<2>{
%		\begin{center}
%			\includegraphics[scale=.18]{Figures/SimulEta1FSex.pdf}
%			\includegraphics[scale=.18]{Figures/SimulEta1MSex.pdf}
%			\includegraphics[scale=.18]{Figures/SimulDelta1Sex.pdf}
%			\includegraphics[scale=.18]{Figures/Simuls1Sex.pdf}
%		\end{center}
%	}
%	\only<3>{
%		\begin{center}
%			\includegraphics[scale=.18]{Figures/SimulEta2FSex.pdf}
%			\includegraphics[scale=.18]{Figures/SimulEta2MSex.pdf}
%			\includegraphics[scale=.18]{Figures/SimulDelta2Sex.pdf}
%			\includegraphics[scale=.18]{Figures/Simuls2Sex.pdf}
%		\end{center}
%	}
%	\only<4>{
%		\begin{center}
%			\includegraphics[scale=.18]{Figures/SimulEta3FSex.pdf}
%			\includegraphics[scale=.18]{Figures/SimulEta3MSex.pdf}
%			\includegraphics[scale=.18]{Figures/SimulDelta3Sex.pdf}
%			\includegraphics[scale=.18]{Figures/Simuls3Sex.pdf}
%		\end{center}
%	}
%\end{frame}
%
%
%\begin{frame}[fragile]\frametitle{A schematic illustration: model \emph{Saturated}}
%	\vspace{-0.2cm}
%\only<1>{
%	\begin{center}
%		\includegraphics[scale=.18]{Figures/SimulEta3FSexSat.pdf}
%		\includegraphics[scale=.18]{Figures/SimulEta3MSexSat.pdf}
%		\includegraphics[scale=.18]{Figures/SimulDelta3SexSat.pdf}
%		\includegraphics[scale=.18]{Figures/Simuls3SexSat.pdf}
%	\end{center}
%}
%\end{frame}
%
%
%\begin{frame}[fragile]\frametitle{Actual data illustration 1: France (different scaling factor)}
%	\vspace{-0.2cm}
%	\only<1>{
%		\begin{center}
%			\includegraphics[scale=.26]{Figures/FranceEtaFSexStra.pdf}
%			\includegraphics[scale=.26]{Figures/FranceEtaMSexStra.pdf}
%			
%			\begin{footnotesize}
%				Fitted log-mortality from \emph{Stratified} model
%			\end{footnotesize}
%			
%		\end{center}
%	}
%	\only<2>{
%	\begin{center}
%		\includegraphics[scale=.26]{Figures/FranceDeltaSexStra.pdf}
%		
%		\begin{footnotesize}
%			Parameters from \emph{Stratified} model
%		\end{footnotesize}
%		
%	\end{center}
%	}
%	\only<3>{
%	\begin{center}
%		\includegraphics[scale=.26]{Figures/FrancesSexCom.pdf}
%		\includegraphics[scale=.26]{Figures/FranceDeltaSexCom.pdf}
%		
%		\begin{footnotesize}
%			Parameters from \emph{Common} model
%		\end{footnotesize}
%		
%	\end{center}
%}
%	\only<4>{
%	\begin{center}
%		\includegraphics[scale=.26]{Figures/FrancesSexComSat.pdf}
%		\includegraphics[scale=.26]{Figures/FranceDeltaSexComSat.pdf}
%		
%		\begin{footnotesize}
%			Parameters from \emph{Common} and \emph{Saturated} model
%		\end{footnotesize}
%		
%	\end{center}
%}
%\end{frame}
%
%
%
%
%\begin{frame}[fragile]\frametitle{Actual data illustration 2: Peru (middle-age hump)}
%	\vspace{-0.2cm}
%\only<1>{
%	\begin{center}
%		\includegraphics[scale=.17]{Figures/PeruEtaFSexSat.pdf}
%		\includegraphics[scale=.17]{Figures/PeruEtaMSexSat.pdf}
%		
%		\includegraphics[scale=.17]{Figures/PerusSexSat.pdf}
%	\includegraphics[scale=.17]{Figures/PeruDeltaSexSat.pdf}
%		
%		\begin{footnotesize}
%			Fitted log-mortality and parameters from \emph{Saturated} model
%		\end{footnotesize}
%		
%	\end{center}
%}
%\only<2>{
%	\begin{center}
%		\includegraphics[scale=.17]{Figures/PeruEtaFSexSat.pdf}
%		\includegraphics[scale=.17]{Figures/PeruEtaMSexSat.pdf}
%		
%		\includegraphics[scale=.17]{Figures/PerusSexSat.pdf}
%		\includegraphics[scale=.17]{Figures/PeruDeltaSexSatCom.pdf}
%		
%		\begin{footnotesize}
%			Fitted log-mortality and parameters from \emph{Saturated} model. $\delta(x)$ and $c$ parameters from the \emph{Common} are plotted along\\
%		\end{footnotesize}
%		
%	\end{center}
%}
%
%\end{frame}
%
%
%
%
%
%\begin{frame}[fragile]\frametitle{Actual data illustration 2: Peru (middle-age hump)}
%	\vspace{-0.2cm}
%	\only<1>{
%		\begin{center}
%			\includegraphics[scale=.26]{Figures/DeltaFAll.pdf}
%			\includegraphics[scale=.26]{Figures/DeltaMAll.pdf}
%			
%			\begin{footnotesize}
%				$\delta(x)$ from \emph{Saturated} model by sex
%			\end{footnotesize}
%			
%		\end{center}
%	}
%	
%\end{frame}
%
%
%
%
%\begin{frame}[fragile]\frametitle{Sex-specific age-dependent component $e^{\delta(x)}$}
%	\only<1>{
%		\vspace{-0.7cm}
%		\begin{center}
%			\includegraphics[scale=.38]{Figures/ExpDeltaGoodPop.pdf}		
%		\end{center}
%	}
%\end{frame}
%
%
%
%\begin{frame}[fragile]\frametitle{Sex age-factor $s(x)$}
%	\only<1>{
%		\vspace{-0.7cm}
%		\begin{center}
%			\includegraphics[scale=.38]{Figures/sGoodPop.pdf}		
%		\end{center}
%	}
%\end{frame}
%
%%
%%
%%\begin{frame}[fragile]\frametitle{Actual data illustration 3: Mongolia (clear data issues)}
%%	\only<1>{
%%		\begin{center}
%%			\includegraphics[scale=.26]{Figures/MongoliaEta.pdf}
%%			\includegraphics[scale=.26]{Figures/MongoliaDelta.pdf}
%%		\end{center}
%%	}
%%\end{frame}
%%
%%\begin{frame}[fragile]\frametitle{Actual data illustration 4: Ireland (a good 2020?)}
%%	\only<1>{
%%		\begin{center}
%%			\includegraphics[scale=.26]{Figures/IrelandEta.pdf}
%%			\includegraphics[scale=.26]{Figures/IrelandDelta.pdf}
%%		\end{center}
%%	}
%%\end{frame}
%%
%%\begin{frame}[fragile]\frametitle{World Map with $e^c$ (relative risk scaling factor)}
%%	\hspace{-2cm}
%%			\includegraphics[scale=0.8]{Figures/scale_factor_map}
%%\end{frame}
%%
%%
%%\begin{frame}[fragile]\frametitle{Age-dependent component $\delta(x)$}
%%	\only<1>{
%%		\begin{center}
%%			\includegraphics[scale=.3]{Figures/DeltaAll.pdf}
%%			
%%			\begin{scriptsize}
%%				Parameter $\delta$ for all available data
%%			\end{scriptsize}
%%			
%%		\end{center}
%%	}
%%
%%	\only<2>{
%%	\begin{center}
%%		\includegraphics[scale=.3]{Figures/DeltaDataIssue.pdf}
%%		
%%		\begin{scriptsize}
%%			Parameter $\delta$ for populations with clear data issue or peculiar patterns
%%		\end{scriptsize}
%%		
%%	\end{center}
%%}
%%	\only<3>{
%%	\begin{center}
%%		\includegraphics[scale=.3]{Figures/DeltaOther.pdf}
%%		
%%		\begin{scriptsize}
%%			Parameter $\delta$ for populations without clear data issue or peculiar patterns
%%		\end{scriptsize}
%%		
%%	\end{center}
%%}
%%	\only<4>{
%%	\begin{center}
%%		\includegraphics[scale=.3]{Figures/DeltaOtherNames.pdf}
%%		
%%		\begin{scriptsize}
%%			Parameter $\delta$ for populations without clear data issue or peculiar patterns
%%		\end{scriptsize}
%%		
%%	\end{center}
%%
%%}
%%\end{frame}
%%\begin{frame}[fragile]\frametitle{Age-dependent component $e^{\delta(x)}$}
%%	\only<1>{
%%		\begin{center}
%%			\includegraphics[scale=.3]{Figures/ExpDeltaOtherNames.pdf}
%%			
%%			\begin{scriptsize}
%%				Parameter $e^\delta$ for populations without clear data issue or peculiar patterns
%%			\end{scriptsize}
%%			
%%		\end{center}
%%	}
%%	\only<2>{
%%		\vspace{-1cm}
%%		\begin{center}
%%			\includegraphics[scale=.38]{Figures/ExpDeltaOtherNamesAll.pdf}		
%%		\end{center}
%%	}
%%\end{frame}
%%
%%
%%
%%\begin{frame}[fragile]\frametitle{An attempt to cluster $\delta(x)$}
%%	\only<1>{
%%		\vspace{-0.9cm}
%%		\begin{center}
%%			\includegraphics[scale=.33]{Figures/ClusterDeltaDendo.pdf}		
%%		\end{center}
%%	}
%%\only<2>{
%%	\hspace{-2cm}
%%	\includegraphics[scale=0.8]{Figures/cluster_map}
%%}
%%\only<3>{
%%	\vspace{-0.1cm}
%%	\begin{center}
%%		\includegraphics[scale=.33]{Figures/ClusterDelta.pdf}		
%%	\end{center}
%%}
%%\only<4>{
%%	\vspace{-0.1cm}
%%	\begin{center}
%%		\includegraphics[scale=.33]{Figures/ClusterDeltaCenter.pdf}		
%%	\end{center}
%%}
%%\end{frame}


\end{document}


%%% -----------------------------------------------------  %%
%%% -----------------------------------------------------  %%
%\begin{frame}[fragile]\frametitle{Quick overview onuthe used data}
%\vspace{-0.4cm}
%\begin{columns}
%	\begin{column}{3.2cm}
%		\begin{itemize}
%			\item[]<1-> Age-specific death rates (probabilities)
%		\end{itemize}
%	\end{column}
%	\begin{column}{9.5cm}
%		\begin{footnotesize}
%			\begin{itemize}
%				\setlength\itemsep{0.1em}
%				\gooditem[\textbf{+}] standard tools for describing mortality (time-to-event)
%				\item[\textbf{+}] easy to incorporate stochasticity of the phenomenon
%				\pooritem[\textbf{--}] latest longevity processes may be missed
%			\end{itemize}
%		\end{footnotesize}
%	\end{column}
%\end{columns}
%\medskip
%\begin{columns}
%	\begin{column}{3.2cm}
%		\begin{itemize}
%			\item[]<2-> Age-at-death distributions
%		\end{itemize}
%	\end{column}
%	\begin{column}{9.5cm}
%		\onslide<2->{
%			\begin{footnotesize}
%				\begin{itemize}
%					\setlength\itemsep{0.1em}
%					\gooditem[\textbf{+}] latest longevity processes can be described
%					\item[\textbf{+}] alternative interpretations ($\sim$ AFT models)
%					\pooritem[\textbf{--}] often applicable only for adult ages
%				\end{itemize}
%			\end{footnotesize}
%		}
%	\end{column}
%\end{columns}
%\medskip
%\begin{columns}
%	\begin{column}{3.2cm}
%		\begin{itemize}
%			\item[]<3-> Survival improvement
%		\end{itemize}
%	\end{column}
%	\begin{column}{9.5cm}
%		\onslide<3->{
%			\begin{footnotesize}
%				\begin{itemize}
%					\setlength\itemsep{0.1em}
%					\gooditem[\textbf{+}] exploit regularities in mortality derivatives  	
%					\pooritem[\textbf{--}] prior model od the original data necessary
%					\item[\textbf{--}] difficulties in including uncertainty measures
%				\end{itemize}
%			\end{footnotesize}
%		}
%	\end{column}
%\end{columns}
%\medskip
%\begin{columns}
%	\begin{column}{3.2cm}
%		\begin{itemize}
%			\item[]<4-> Summary measures
%		\end{itemize}
%	\end{column}
%	\begin{column}{9.5cm}
%		\onslide<4->{
%			\begin{footnotesize}
%				\begin{itemize}
%					\setlength\itemsep{0.1em}
%					\gooditem[\textbf{+}] straightforward interpretation
%					\item[\textbf{+}] easy to impose prior assumptions
%					\pooritem[\textbf{--}] hard to move to complete age-time mortality profiles
%				\end{itemize}
%			\end{footnotesize}
%		}
%	\end{column}
%\end{columns}
%
%
%\end{frame}
%
%
%
%
%
%\begin{frame}[fragile]\frametitle{Placing some reference}
%
%\vspace{-1cm}
%{\fontsize{6pt}{9}\selectfont 
%	\begin{tabular}{p{8cm}p{0.5\textwidth}}
%		&
%		\begin{itemize}
%			\itemsep-0.3em 
%			\item[] {\color{Blue}Single population}
%			\item[] {\color{Green}More populations}
%			\item[] {\color{Red}Strata of population}
%		\end{itemize} 
%	\end{tabular}
%	\vspace{-1cm}
%	%\begin{tiny}
%	\begin{table}[]
%		\setlength\tabcolsep{1pt}
%		\hspace*{-0.8cm} 
%		\begin{tabular}{ccllll}
%			\multicolumn{2}{l}{} & \multicolumn{4}{c}{Data} \\ \cline{3-6} 
%			\multicolumn{2}{l|}{} & \multicolumn{1}{c|}{Rates/Probabilities} & \multicolumn{1}{c|}{Age-at-death distributions} & \multicolumn{1}{c|}{Rates-of-change} & \multicolumn{1}{c}{Summary measures} \\ \cline{2-6}
%			\multicolumn{1}{c|}{} & \multicolumn{1}{c|}{\cellcolor[HTML]{C0C0C0}Parametric} & \multicolumn{1}{l|}{\cellcolor[HTML]{C0C0C0}\begin{tabular}[c]{@{}l@{}}{\color{Blue}Classic models (1825-)}\\ {\color{Blue}Cairn et al. (2006)}\\ {\color{Red}Alai et al. (2018)}\end{tabular}} & \multicolumn{1}{l|}{\cellcolor[HTML]{C0C0C0}{\color{Blue}Janseen \& De Beer (2016)}} & \multicolumn{1}{l|}{\cellcolor[HTML]{C0C0C0}{\color{Blue}Haberman \& Renshaw (2012)}} & \cellcolor[HTML]{C0C0C0}{\color{Green}Raftery et al. (2013)} \\
%			\multicolumn{1}{c|}{} & \multicolumn{1}{l|}{\begin{tabular}[c]{@{}c@{}}Principal\\ component\end{tabular}} & \multicolumn{1}{l|}{\begin{tabular}[c]{@{}l@{}}{\color{Blue}Lee \& Carter (1992) + variants}\\ {\color{Green}Li \& Lee (2005)}\\ {\color{Red}Li et al. (2019)}\end{tabular}} & \multicolumn{1}{l|}{\begin{tabular}[c]{@{}l@{}}{\color{Red}Oeppen (2008)}\\ {\color{Green}Bergeron-Boucher et al. (2017)}\\ {\color{Red}Kj\ae rgaard et al. (2019)}\end{tabular}} & \multicolumn{1}{l|}{\begin{tabular}[c]{@{}l@{}}{\color{Blue}Mitchell et al. (2013)}\\ {\color{Green}Bohk \& Rau (2017)}\end{tabular}} &  \\
%			\multicolumn{1}{c|}{} & \multicolumn{1}{c|}{\cellcolor[HTML]{C0C0C0}Relational} & \multicolumn{1}{l|}{\cellcolor[HTML]{C0C0C0}{\color{Blue}De Beer (2012)}} & \multicolumn{1}{l|}{\cellcolor[HTML]{C0C0C0}{\color{Blue}Basellini \& Camarda (2019)}} & \multicolumn{1}{l|}{\cellcolor[HTML]{C0C0C0}} & \cellcolor[HTML]{C0C0C0}{\color{Blue}Torri \& Vaupel (2012)} \\
%			%		\rot{structure}&
%			\multicolumn{1}{c|}{\multirow{-8}{*}{\rotatebox[origin=c]{90}{structure}}} &
%			\multicolumn{1}{l|}{\begin{tabular}[c]{@{}c@{}}Non-\\ parametric\end{tabular}} & \multicolumn{1}{l|}{\begin{tabular}[c]{@{}l@{}}{\color{Blue}Currie et al. (2004)}\\ {\color{Blue}Camarda (2019)}\end{tabular}} & \multicolumn{1}{l|}{} & \multicolumn{1}{l|}{} & 
%		\end{tabular}
%	\end{table}
%}
%
%
%%\end{tiny} \parbox[t]{1mm}{\multirow{-7}{*}{\rotatebox[origin=c]{90}{structure}}}
%\end{frame}
%
%

%
%
%%
%%%% -----------------------------------------------------  %%
%%%% -----------------------------------------------------  %%
%%\begin{frame}[fragile]\frametitle{Outcomes from plain approaches}
%%\vspace{-0.1cm}
%%\begin{small}
%%  \begin{itemize}
%%\setlength\itemsep{0em}
%%  \item<1-> USA, males, ages 0-105, years 1960-2016, forecast years 2017-2050
%%  \end{itemize}
%%\end{small}
%%    \only<1>{
%%\vspace{-0.3cm}
%%      \begin{center}
%%        \includegraphics[scale=.30]{Figures/ExampleActualYEAR.pdf}
%%      \end{center}
%%    }
%%    \only<2>{
%%\vspace{-0.3cm}
%%      \begin{center}
%%        \includegraphics[scale=.30]{Figures/ExampleForecastYEAR.pdf}
%%      \end{center}
%%    }
%%\vspace{-0.2cm}
%%\begin{tiny}
%%\begin{itemize}
%%\setlength\itemsep{0em}
%%\item[] - Source: Human Mortality Database
%%\item[] - Delwarde et al.~(2007) for the smooth Lee-Carter
%%\item[] - Currie et al.~(2004) for the two-dimensional $P$-splines, 
%%\end{itemize}
%%
%%\end{tiny}
%%
%%\end{frame}
%%
%%\begin{frame}[fragile]\frametitle{The idea}
%%\begin{columns}
%%\begin{column}{5cm}
%%    \begin{block}{\textbf{$P$-splines:}}
%%\begin{itemize}
%%\gooditem flexibility (good fit)
%%\pooritem blind adherence extrapolation
%%\end{itemize}
%%    \end{block}
%%\end{column}
%%\begin{column}{5cm}
%%    \begin{block}{\textbf{Lee-Carter:}}
%%\begin{itemize}
%%\pooritem rigidity (bad fit)
%%\gooditem past development drives future mortality
%%\end{itemize}
%%    \end{block}
%%\end{column}
%%\end{columns}
%%\bigskip
%%\bigskip
%%\onslide<2->{
%%    \begin{alertblock}{\centering \textbf{Constrained $P$-splines, $CP$-splines:}}
%%Incorporate into $P$-splines demographic information
%%\begin{itemize}
%%\gooditem flexibility (good fit)
%%\gooditem future development based on prior demographic knowledge
%%\end{itemize}
%%\end{alertblock}
%%}
%%\end{frame}
%%
%\section[Lee-Carter]{The benchmark Lee-Carter model}
%
%\begin{frame}\frametitle{The model}
%\begin{itemize}
%	\item<1-> Lee and Carter (1992) suggested to model directly the
%	log-death rates:
%	\begin{equation*}
%	\ln(m_{ij}) =  {\color{Red}\alpha_{i}} +  {\color{Blue}\beta_{i}} \,  {\color{Green}\kappa_{j}} +
%	\epsilon_{ij}
%	\end{equation*}\begin{itemize}
%		\item ${\color{Red}\alpha_{i}}$, ${\color{Blue}\beta_{i}} $ and ${\color{Green}\kappa_{j}}$ are vectors of parameters
%		\item $\epsilon_{ij}$ is a matrix of errors
%	\end{itemize}
%	\medskip
%	\item<2-> Interpretation of the parameters:
%	\begin{itemize}
%		\item ${\color{Red}\alpha_{i}}$ is the general shape of the log-mortality at age $i$
%		\medskip
%		\item ${\color{Green}\kappa_{j}}$ represents the time trend
%		\medskip
%		\item ${\color{Blue}\beta_{i}}$ indicates the sensitivity of the log-mortality at age $i$ to
%		variations in the time index
%	\end{itemize}
%\end{itemize}
%\end{frame}
%
%%\begin{frame}\frametitle{Lee-Carter: schematic view}
%%\begin{small}
%%	\begin{itemize}
%%		\item<1->[]
%%		\begin{equation*}
%%		\left ( \begin{array}{cccc}
%%		\ln(m_{0,1960}) & \ln(m_{0,1961}) & \ldots & \ln(m_{0,2015})\\
%%		\ln(m_{1,1960}) & \ln(m_{1,1961}) & \ldots & \ln(m_{1,2015})\\
%%		\ln(m_{2,1960}) & \ln(m_{2,1961}) & \ldots & \ln(m_{2,2015})\\
%%		\vdots & \vdots & \ddots & \vdots \\
%%		\ln(m_{105,1960}) & \ln(m_{105,1961}) & \ldots & \ln(m_{105,2015})\\
%%		\end{array} \right ) \simeq
%%		\end{equation*}
%%		\begin{equation*}
%%		\left ( \begin{array}{c}
%%		{\color{Red}\alpha_{0}}\\
%%		{\color{Red}\alpha_{1}}\\
%%		{\color{Red}\alpha_{2}}\\
%%		{\color{Red}\vdots}\\
%%		{\color{Red}\alpha_{105}}\\
%%		\end{array} \right ) +
%%		\left ( \begin{array}{c}
%%		{\color{Blue}\beta_{0}}\\
%%		{\color{Blue}\beta_{1}}\\
%%		{\color{Blue}\beta_{2}}\\
%%		{\color{Blue}\vdots}\\
%%		{\color{Blue}\beta_{105}}\\
%%		\end{array} \right ) \,
%%		\left (
%%		\begin{array}{cccc}
%%		{\color{Green}\kappa_{1960}} & {\color{Green}\kappa_{1961}} & {\color{Green}\ldots} & {\color{Green}\kappa_{2015}}
%%		\end{array} \right )
%%		\end{equation*}
%%		\item<2->[]
%%		\begin{equation*}
%%		\underbrace{56}_{\mbox{years}} \times
%%		\underbrace{106}_{\mbox{ages}} = \underbrace{5936}_{\mbox{cells}}
%%		\simeq \underbrace{106}_{{\color{Red}\alpha_{i}}} + \underbrace{106}_{{\color{Blue}\beta_{i}}}
%%		+ \underbrace{56}_{{\color{Green}\kappa_{j}}} -
%%		\underbrace{2}_{\mbox{constraints}} = \underbrace{266}_{\mbox{parameters}}
%%		\end{equation*}
%%	\end{itemize}
%%\end{small}
%%\end{frame}
%
%\begin{frame}\frametitle{Estimation procedure}
%\begin{itemize}
%	\item<1-> Lee and Carter (1992) used Singular Value Decomposition
%%	\item<2-> Wilmoth (1993) suggested to fit the LC by Weighed Least-Squares (WLS)
%%	estimation
%	\item<2-> Brouhns et al.~(2002) suggested a Poisson log-bilinear model
%	for the estimated the LC
%
%\begin{itemize}
%\item Data: deaths and exposures over age and years: $Y_{ij}$ and $E_{ij}$
%\item Assumption: $Y_{ij} \sim \mathcal{P}(\mu_{ij} E_{ij})$ with $\mu_{ij} = \exp({\color{Red}\alpha_{i}} + {\color{Blue}\beta_{i}} \cdot {\color{Green}\kappa_{j}})$
%\item Maximization of the likelihood:
%$$
%\ln\mathcal{L}\left({\color{Red}\bm{\alpha}}, {\color{Blue}\bm{\beta}}, {\color{Green}\bm{\kappa}} | \, \bm{Y} , \bm{E}\right) \propto \sum_{i,j} \left[ Y_{ij} \, \ln \left ( \mu_{ij}  \right ) - E_{ij}\, \mu_{ij} \right]  \, . 
%$$
%\end{itemize}
%	\item<3-> Delwarde et al.~(2007) used penalized likelihood to enforce smoothness of parameter vectors:
%	\begin{small}
%	$$
%	\ln\mathcal{L}^{P}\left(\cdot\right) = \ln\mathcal{L}\left(\cdot\right) - \frac{1}{2} \,\lambda_{\alpha} \, {\color{Red}\bm{\alpha}}' \,\bm{D}'\bm{D} \, {\color{Red}\bm{\alpha}} - \frac{1}{2} \,\lambda_{\beta} \, {\color{Blue}\bm{\beta}}' \,\bm{D}'\bm{D} \, {\color{Blue}\bm{\beta}}
%	$$
%	\end{small}where $\lambda_{\alpha}$ and $\lambda_{\beta}$ control smoothness of ${\color{Red}\bm{\alpha}}$ and ${\color{Blue}\bm{\beta}}$ and $\bm{D}$ is the second order difference matrix
%\end{itemize}
%\end{frame}
%
%
%\begin{frame}[fragile]\frametitle{Portuguese female mortality in the light of Lee-Carter}
%\vspace{-0.4cm}
%\only<1>{
%	\begin{center}
%		\includegraphics[scale=.19]{Figures/LCalpha.pdf}
%		\includegraphics[scale=.19]{Figures/LCbeta.pdf}
%		\includegraphics[scale=.19]{Figures/LCkappa.pdf}
%	\end{center}
%}
%\end{frame}
%
%
%%\begin{frame}[fragile]\frametitle{Forecasting with the Lee-Carter}
%%\begin{itemize}
%%	\item Only ${\color{Green}\kappa_{j}}$ depends on time $\Rightarrow$ to forecast the whole surface:
%%	\begin{enumerate}
%%		\item fix ${\color{Red}\alpha_{i}}$ and ${\color{Blue}\beta_{i}}$
%%		\item extrapolate a single time-series (${\color{Green}\kappa_{j}}$), commonly rather linear
%%		\item given future ${\color{Green}\kappa_{j}}$, we reconstruct all age-year development
%%	\end{enumerate}
%%\bigskip
%%	\only<1>{
%%		\includegraphics[scale=0.12]{./Figures/Empty}
%%	}
%%	\item<2-> Schematically:
%%\end{itemize}
%%\only<2>{
%%	\begin{small}
%%		\begin{equation*}
%%		\left ( \begin{array}{cccccc}
%%		\ln(m_{0,1960}) & \ldots & \ln(m_{0,2015}) \\
%%		\vdots  & \ddots & \vdots \\
%%		\ln(m_{105,1960}) & \ldots & \ln(m_{105,2015})\\
%%		\end{array} \right ) \simeq
%%		\end{equation*}
%%		\begin{equation*}
%%		\left ( \begin{array}{c}
%%		{\color{Red}\alpha_{0}}\\
%%		{\color{Red}\vdots}\\
%%		{\color{Red}\alpha_{105}}\\
%%		\end{array} \right ) +
%%		\left ( \begin{array}{c}
%%		{\color{Blue}\beta_{0}}\\
%%		{\color{Blue}\vdots}\\
%%		{\color{Blue}\beta_{105}}\\
%%		\end{array} \right ) \,
%%		\left (
%%		\begin{array}{ccc}
%%		{\color{Green}\kappa_{1960}} & {\color{Green}\ldots} & {\color{Green}\kappa_{2015}}
%%		\end{array} \right )
%%		\end{equation*}
%%	\end{small}
%%}
%%%
%%\only<3>{
%%	\begin{small}
%%		\begin{equation*}
%%		\left ( \begin{array}{cccccc}
%%		\ln(m_{0,1960}) & \ldots & \ln(m_{0,2015}) & {\color{orange}\ldots} & {\color{orange}\ldots} & {\color{orange}\ldots}\\
%%		\vdots  & \ddots & \vdots & {\color{orange}\vdots} & {\color{orange}\ddots} & {\color{orange}\vdots}\\
%%		\ln(m_{105,1960}) & \ldots & \ln(m_{105,2015})& {\color{orange}\ldots} & {\color{orange}\ldots} & {\color{orange}\ldots}\\
%%		\end{array} \right ) \simeq
%%		\end{equation*}
%%		\begin{equation*}
%%		\left ( \begin{array}{c}
%%		{\color{Red}\alpha_{0}}\\
%%		{\color{Red}\vdots}\\
%%		{\color{Red}\alpha_{105}}\\
%%		\end{array} \right ) +
%%		\left ( \begin{array}{c}
%%		{\color{Blue}\beta_{0}}\\
%%		{\color{Blue}\vdots}\\
%%		{\color{Blue}\beta_{105}}\\
%%		\end{array} \right ) \,
%%		\left (
%%		\begin{array}{cccccc}
%%		{\color{Green}\kappa_{1960}} & {\color{Green}\ldots} & {\color{Green}\kappa_{2015}} & {\color{orange}\kappa_{2016}} & {\color{orange}\ldots} & {\color{orange}\kappa_{2050}}
%%		\end{array} \right )
%%		\end{equation*}
%%	\end{small}
%%}	
%%\only<4>{
%%	\begin{small}
%%		\begin{equation*}
%%		\left ( \begin{array}{cccccc}
%%		\ln(m_{0,1960}) & \ldots & \ln(m_{0,2015}) & {\color{orange}\ln(m_{0,2016})} & {\color{orange}\ldots} & {\color{orange}\ln(m_{0,2050})}\\
%%		\vdots  & \ddots & \vdots & {\color{orange}\vdots} & {\color{orange}\ddots} & {\color{orange}\vdots}\\
%%		\ln(m_{105,1960}) & \ldots & \ln(m_{105,2015})& {\color{orange}\ln(m_{105,2016})} & {\color{orange}\ldots} & {\color{orange}\ln(m_{105,2050})}\\
%%		\end{array} \right ) \simeq
%%		\end{equation*}
%%		\begin{equation*}
%%		\left ( \begin{array}{c}
%%		{\color{Red}\alpha_{0}}\\
%%		{\color{Red}\vdots}\\
%%		{\color{Red}\alpha_{105}}\\
%%		\end{array} \right ) +
%%		\left ( \begin{array}{c}
%%		{\color{Blue}\beta_{0}}\\
%%		{\color{Blue}\vdots}\\
%%		{\color{Blue}\beta_{105}}\\
%%		\end{array} \right ) \,
%%		\left (
%%		\begin{array}{cccccc}
%%		{\color{Green}\kappa_{1960}} & {\color{Green}\ldots} & {\color{Green}\kappa_{2015}} & {\color{orange}\kappa_{2016}} & {\color{orange}\ldots} & {\color{orange}\kappa_{2050}}
%%		\end{array} \right )
%%		\end{equation*}
%%	\end{small}
%%}	
%%\end{frame}
%
%\begin{frame}[fragile]\frametitle{Future Portuguese female mortality \only<1,2>{: Lee-Carter ${\color{Green}\kappa_{j}}$}\only<3,4>{: log-rates}}
%\vspace{-0.1cm}
%\only<1>{
%	\begin{center}
%		\includegraphics[scale=.26]{Figures/LCkappaFor0.pdf}
%	\end{center}
%}
%\only<2>{
%	\begin{center}
%		\includegraphics[scale=.26]{Figures/LCkappaFor.pdf}
%	\end{center}
%}
%\only<3>{
%	\vspace{-0.4cm}
%	\begin{center}
%		\includegraphics[scale=.32]{Figures/ExampleLCYEAR.pdf}
%	\end{center}
%}
%\only<4>{
%	\vspace{-0.4cm}
%	\begin{center}
%		\includegraphics[scale=.32]{Figures/ExampleLCforYEAR.pdf}
%	\end{center}
%}
%\end{frame}
%
%\begin{frame}[fragile]\frametitle{Much more about the Lee-Carter}
%\begin{itemize}
%	\item Issues and solutions:
%	\medskip
%	
%	\begin{tabular}{ll}
%		Fix $\beta_{i}$: & {\scriptsize Li et al. (2013)}\\
%		Smoothness: &  {\scriptsize Hyndman and Ullah (2007); Currie (2014)}\\
%		Accounting for uncertainty: & {\scriptsize Brouhns et al. (2005); Koissi et al. (2006)}
%	\end{tabular}
%
%\bigskip
%\item Extensions:
%\medskip
%
%\begin{tabular}{ll}
%	Low quality data: & {\scriptsize Li et al. (2007)}\\
%More time-indexes: & {\scriptsize Hyndman and Ullah (2007)}\\
%Coherent forecast: & {\scriptsize Li and Lee (2005); Hyndman et al. (2013)}\\
%Cohort effects: & {\scriptsize Renshaw and Haberman (2006)}\\
%Bayesian estimation: & {\scriptsize Czado et al. (2005); Girosi and King (2008)}\\
%Age-pattern decomposition: & {\scriptsize Camarda and Basellini (2020)}
%\end{tabular}
%
%\end{itemize}
%
%
%\end{frame}
%
%
%\section[A non-parametric approach$\qquad\qquad$]{The statistical alternative: a non-parametric approach}
%
%%\begin{frame}[fragile]\frametitle{Data structure \& basic model}
%%\vspace{-0.5cm}
%%\begin{columns}
%%\begin{column}{0.5\textwidth}
%%%\vspace{-0.5cm}
%%\begin{center}
%%{\color{Blue}Observed data}\\\medskip
%%\includegraphics[scale=.2]{Figures/DataMort.pdf}
%%%        \includegraphics[scale=.2]{Figures/SchemeDEA.pdf}
%%%        \includegraphics[scale=.04]{Figures/SchemeWHITE.pdf}
%%%        \includegraphics[scale=.2]{Figures/SchemeEXP.pdf}
%%\end{center}
%%\onslide<2->{
%%	\vspace{-0.2cm}
%%\centering {\color{Green}Assumption:}
%%\vspace{-0.4cm}
%%$$
%%y_{ij} \sim \mathcal{P}(e_{ij} \,\mu_{ij})\, 
%%$$
%%}
%%\end{column}
%%% \vrule{}
%%\begin{column}{0.65\textwidth}
%%% \vspace{1cm}
%%\onslide<3->{
%%\centering{\color{Red}Model:}
%%}
%%\begin{itemize}
%%\setlength\itemsep{0em}
%%\item<3->[] $\bm{y} = \verb"vec"(\bm{Y})$
%%\item<3->[] $\bm{e} = \verb"vec"(\bm{E})$
%%\end{itemize}
%%\onslide<3->{
%%\vspace{-0.3cm}
%%\begin{eqnarray*}
%%\ln(\bm{y}) = \ln(\bm{e})+ \ln(\bm{\mu}) &=& \ln(\bm{e}) + \bm{\eta}\\
%%&=& \ln(\bm{e}) + \bm{B}\,\bm{\alpha}
%%\end{eqnarray*}
%%}
%%\vspace{-.7cm}
%%\begin{itemize}
%%\setlength\itemsep{0em}
%%\item<3->[] $\bm{B} = {\color{Blue}\bm{B}_{t_{1}}} \otimes {\color{Blue}\bm{B}_{a}}$
%%\item<3->[] $\bm{\alpha}$: penalized coefficients
%%\end{itemize}
%%\onslide<4->{
%%%\vspace{-.2cm}
%%\includegraphics[scale=.35]{Figures/KronBsplinesINK.pdf}
%%}
%%\end{column}
%%\end{columns}
%%\begin{small}
%%\vspace{-.5cm}
%%\begin{itemize}
%%\setlength\itemsep{0em}
%%\item<5-> estimated by a penalized iterative weighted least-squares algorithm
%%\end{itemize}
%%\end{small}
%%
%%\end{frame}
%
%
%
%\begin{frame}[fragile]\frametitle{Forecasting with $P$-splines (a missing value problem)}
%\begin{columns}
%	\begin{column}{4cm}
%		$\ln(\bm{y}) = \ln(\bm{e}) + \bm{B}\,\bm{\alpha}$
%	\end{column}
%	\begin{column}{5cm}
%\vspace{-.7cm}
%\begin{itemize}
%	\setlength\itemsep{0em}
%	\item<1->[] $\bm{B} = {\color{Blue}\bm{B}_{t_{1}}} \otimes {\color{Blue}\bm{B}_{a}}$
%	\item<1->[] $\bm{\alpha}$: penalized coefficients
%\end{itemize}
%\end{column}
%
%\end{columns}
%
%\smallskip
%\begin{itemize}
%\item<2-> Data are augmented with arbitrary future values and we define a weight matrix (Currie et al.~2004)
%\item<3-> $B$-spline bases are augmented too: $\bm{B} = [{\color{Blue}\bm{B}_{t_{1}}}:{\color{Red}\bm{B}_{t_{2}}}] \otimes {\color{Blue}\bm{B}_{a}}$
%\end{itemize}
%\begin{itemize}
%\pooritem<4-> No information about mortality structure is used
%\end{itemize}
%
%\begin{columns}
%\begin{column}{0.55\textwidth}
%\onslide<2->{
%\includegraphics[scale=.12]{Figures/SchemeDEAfor.pdf}
%        % \includegraphics[scale=.02]{Figures/SchemeWHITE.pdf}
%\includegraphics[scale=.12]{Figures/SchemeEXPfor.pdf}
%        % \includegraphics[scale=.02]{Figures/SchemeWHITE.pdf}
%\includegraphics[scale=.12]{Figures/SchemeWEIfor.pdf}
%}
%\end{column}
%\begin{column}{0.6\textwidth}
%\only<1,2>{\includegraphics[scale=.4]{Figures/KronBsplinesINK.pdf}}
%\only<3->{\includegraphics[scale=.4]{Figures/KronBsplinesFORINK.pdf}}
%\end{column}
%\end{columns}
%
%
%\end{frame}
%%
%%% -----------------------------------------------------  %%
%%% -----------------------------------------------------  %%
%%\section{$CP$-splines}
%%
%%%% -----------------------------------------------------  %%
%%%% -----------------------------------------------------  %%
%\begin{frame}\frametitle{Observing mortality patterns (over ages)}
%\vspace{-0.3cm}
%\begin{itemize}
%\item<1-> Over ages we observe certain regularities in mortality shape
%\vspace{0.3cm}
%\only<1>{
%	\begin{center}
%\includegraphics[scale=.24]{Figures/ExampleAgeShape.pdf}
%\includegraphics[scale=.24]{Figures/ExampleAgeDerivEmpty.pdf}
%\end{center}
%}
%\only<2>{
%	\begin{center}
%\includegraphics[scale=.24]{Figures/ExampleAgeShape.pdf}
%\includegraphics[scale=.24]{Figures/ExampleAgeDeriv.pdf}
%\end{center}
%}
%\vspace{0.1cm}
%\item<2-> Interested solely in shapes (regardless their levels), these
%regularities are condensed in the (relative) derivatives
%\end{itemize}
%\end{frame}
%
%%% -----------------------------------------------------  %%
%%% -----------------------------------------------------  %%
%\begin{frame}[fragile]\frametitle{Observing mortality patterns (over years)}
%\vspace{-0.3cm}
%  \begin{itemize}
%  \item<1-> Variations over time are larger than over age
%  \item<2-> We can still observe regularities for each age
%\only<1>{
%		\begin{center}
%        \includegraphics[scale=.18]{Figures/EmptyYear.pdf}
%
%        \includegraphics[scale=.18]{Figures/EmptyYear.pdf}
%        	\end{center}
%    }
%\only<2>{
%		\begin{center}
%        \includegraphics[scale=.18]{Figures/ExampleYearShape.pdf}
%
%        \includegraphics[scale=.18]{Figures/EmptyYear.pdf}
%                	\end{center}
%    }
%\only<3>{
%		\begin{center}
%        \includegraphics[scale=.18]{Figures/ExampleYearShape1.pdf}
%
%        \includegraphics[scale=.18]{Figures/ExampleYearDeriv.pdf}
%                	\end{center}
%    }
%\only<4>{
%		\begin{center}
%        \includegraphics[scale=.18]{Figures/ExampleYearShape1.pdf}
%
%        \includegraphics[scale=.18]{Figures/ExampleYearDerivCI.pdf}
%                	\end{center}
%    }
%\only<5>{
%		\begin{center}
%\includegraphics[scale=.36]{Figures/ExampleYearDerivCIallAges.pdf}       
%        	\end{center}
%    }
%\item<3-> Interested in age-specific rate-of-change,
%  i.e.~derivatives
%\item<4-> Ignore small fluctuations $\Rightarrow$ we take 50\%
%  confidence interval 
%\end{itemize}
%
%\end{frame}
%
%
% \begin{frame}[fragile]\frametitle{Working on the shape}
% \begin{itemize}
% \item<1-> Disregarding information about shapes seems unreasonable
% \medskip
% \item<2-> Future mortality {\color{Red}\emph{must}} follow data-driven
%   age-profiles and rate-of-change
% \medskip
% \item<3-> We constrain derivatives of future mortality to lay
%   within certain confidence intervals of observed derivatives
% \medskip
% \item<4-> Asymmetric penalties are employed for this purpose:
% \begin{itemize}
% % \item the main idea behind this penalty lays in its local behavior 
% \item within each iteration, whenever current estimations present
%   derivatives (in future years) out of the desired intervals, a
%   penalty intervenes 
% \end{itemize}
% \end{itemize}
%  \onslide<5->{
%  \begin{block}{}
%  	\centering 	\color{Red}{\begin{Large}
%  			$CP$-splines 
%  		\end{Large}} 
%  \end{block}	
%\smallskip
%
%{\scriptsize
%Camarda, C. G. (2019).\\
%Smooth Constrained Mortality Forecasting. \\
%\textit{Demographic Research}. \textbf{41} (38), 1091-1130\\
%}
%}
% \end{frame}
%
%
%
%\begin{frame}[fragile]\frametitle{Future Portuguese female mortality by both approaches}\vspace{-0.1cm}
%\only<1>{
%	\vspace{-0.4cm}
%	\begin{center}
%		\includegraphics[scale=.32]{Figures/ExampleLCCPSforYEAR.pdf}
%	\end{center}
%}
%\only<2>{
%	\vspace{-0.4cm}
%	\begin{center}
%		\includegraphics[scale=.26]{Figures/ExampleE0.pdf}
%		\includegraphics[scale=.26]{Figures/ExampleEDink.pdf}
%	\end{center}
%}
%\only<3>{
%	\begin{center}
%		\animategraphics[autoplay,scale=0.21]{5}{./Figures/ExampleLCforAGEanim}{0}{90}
%		\animategraphics[autoplay,scale=0.21]{5}{./Figures/ExampleCPSforAGEanim}{0}{90}
%	\end{center}
%}
%
%\end{frame}
%
%
%\section*{$\qquad$}
%
%
%
%\begin{frame}[fragile]\frametitle{Take-home messages}
%%\begin{itemize}
%%	\item<1-> blabla
%%\end{itemize}
%\onslide<1->{
%	\begin{block}{\centering \color{Blue}{\textbf{Conclusion:}}}
%		\centering  
%		\color{Red}{\begin{large}
%				Despite its well deserved success, the Lee-Carter model has 
%				\begin{itemize}
%					\item monopolized the field
%					\item intrinsic flaws, which are only partially solved
%				\end{itemize}
%			\end{large}
%		}
%	\end{block}
%}
%\bigskip
%\onslide<2->{
%	\begin{block}{\centering \color{Blue}{\textbf{Corollary:}}}
%		\centering  
%		\color{Red}{\begin{large}
%		New paradigms are necessary \\(are non-parametric approaches possible avenues?)\\
%			\end{large}
%		}
%	\end{block}
%}
%\end{frame}
%
%
%\begin{frame}\frametitle{$\,$}
%\vspace{-0.5cm}
%{\color{Red}
%\begin{center}
%	\begin{tabular}{lll}
%		$\qquad$$\qquad$$\qquad$$\qquad$ $\qquad$$\qquad$&
%		$\qquad$ \\  
%		\hline
%	\end{tabular}
%\end{center}
%}
%
%%
%\begin{center}
%\begin{LARGE}
%	Thanks for your attention.\\
%	\smallskip
%	Comments and questions?
%\end{LARGE}
%\end{center}
%\vspace{-0.7cm}
%{\color{Red}
%\begin{center}
%	\begin{tabular}{lll}
%		$\qquad$$\qquad$$\qquad$$\qquad$ $\qquad$$\qquad$&
%		$\qquad$ \\  
%		\hline
%	\end{tabular}
%\end{center}
%}
%\begin{center}
%\begin{large}
%	\smallskip
%	
%	%\texttt{carlo-giovanni.camarda@ined.fr}\\
%	
%	{\color{Blue}\texttt{sites.google.com/site/carlogiovannicamarda}}
%	
%	
%	\bigskip
%	\bigskip
%	\bigskip
%	
%	
%\end{large}
%\end{center}
%%
%%\begin{tiny}
%%\epigraph{\emph{Null hypothesis is never proved or established, \\but is possibly disproved, \\in the course of experimentation.}}{Ronald Fisher (1935)\\\emph{The Design of Experiments} \\Oliver and Boyd, p.~18}
%%\end{tiny}
%
%\end{frame}
%
%
%
%
%%
%%
%%
%%
%%\section{$P$-spline intermezzo}
%%
%%
%%\begin{frame}\frametitle{Simple least-squares}
%%\begin{itemize}
%%	\item<1-> We have an $n$-vector $\bm{y}$ and covariate(s) $\bm{x}$
%%	\item<2-> We assume $E(y_{i}) = \mu_{i} = \alpha_{0} + \alpha_{1} x_{i}$
%%	\item<3-> We minimize:
%%	\begin{eqnarray*}
%%		S = \sum_{i=1}^{n} \left[ y_{i} - ({\color{Red}\alpha_{0}} + {\color{Green}\alpha_{1} x_{i}}\only<6->{+ {\color{Blue}\alpha_{2} x^{2}_{i}}}\only<7->{+ {\color{Cyan}\alpha_{3} x^{3}_{i}}}\only<8->{+ {\color{violet}\alpha_{4} x^{4}_{i}}})\right]^{2}
%%	\end{eqnarray*}
%%	\item<4-> In matrix:
%%	\onslide<4,5>{
%%		\begin{eqnarray*}
%%			\bm{X} = \left[ \begin{array}{cc}
%%				{\color{Red}1} & {\color{Green}x_{1}} \\
%%				{\color{Red}1} & {\color{Green}x_{2}} \\
%%				{\color{Red}\vdots} & {\color{Green}\vdots} \\
%%				{\color{Red}1} & {\color{Green}x_{n}} \\
%%			\end{array}\right]
%%			\quad \bm{\alpha} = \left[ \begin{array}{c}
%%				{\color{Red}\alpha_{0}} \\
%%				{\color{Green}\alpha_{1}} \\
%%			\end{array}\right]
%%			\quad \bm{\mu} = \bm{X} \bm{\alpha}
%%		\end{eqnarray*}
%%	}
%%	\vspace{-2.5cm}
%%	\onslide<6>{
%%		\begin{eqnarray*} \bm{X} = \left[ \begin{array}{ccc}
%%				{\color{Red}1} & {\color{Green}x_{1}} & {\color{Blue}x^{2}_{1}}\\
%%				{\color{Red}1} & {\color{Green}x_{2}} & {\color{Blue}x^{2}_{2}}\\
%%				{\color{Red}\vdots} & {\color{Green}\vdots} & {\color{Blue}\vdots} \\
%%				{\color{Red}1} & {\color{Green}x_{n}} & {\color{Blue}x^{2}_{n}}
%%			\end{array}\right]
%%			\quad \bm{\alpha} = \left[ \begin{array}{c}
%%				{\color{Red}\alpha_{0}} \\
%%				{\color{Green}\alpha_{1}} \\
%%				{\color{Blue}\alpha_{2}}
%%			\end{array}\right]
%%			\quad , \quad \bm{\mu} = \bm{X} \bm{\alpha}
%%		\end{eqnarray*}
%%	}
%%	\vspace{-2.5cm}
%%	\onslide<7>{
%%		\begin{eqnarray*} \bm{X} = \left[ \begin{array}{cccc}
%%				{\color{Red}1} & {\color{Green}x_{1}} & {\color{Blue}x^{2}_{1}} & {\color{Cyan}x^{3}_{1}}\\
%%				{\color{Red}1} & {\color{Green}x_{2}} & {\color{Blue}x^{2}_{2}} & {\color{Cyan}x^{3}_{2}}\\
%%				{\color{Red}\vdots} & {\color{Green}\vdots} & {\color{Blue}\vdots} & {\color{Cyan}\vdots}\\
%%				{\color{Red}1} & {\color{Green}x_{n}} & {\color{Blue}x^{2}_{n}} & {\color{Cyan}x^{3}_{n}}\\
%%			\end{array}\right]
%%			\quad \bm{\alpha} = \left[ \begin{array}{c}
%%				{\color{Red}\alpha_{0}} \\
%%				{\color{Green}\alpha_{1}} \\
%%				{\color{Blue}\alpha_{2}} \\
%%				{\color{Cyan}\alpha_{3}}
%%			\end{array}\right]
%%			\quad , \quad \bm{\mu} = \bm{X} \bm{\alpha}
%%		\end{eqnarray*}
%%	}
%%	\vspace{-2.5cm}\onslide<8>{
%%		\begin{eqnarray*} \bm{X} = \left[ \begin{array}{ccccc}
%%				{\color{Red}1} & {\color{Green}x_{1}} & {\color{Blue}x^{2}_{1}} & {\color{Cyan}x^{3}_{1}} & {\color{violet}x^{4}_{1}}\\
%%				{\color{Red}1} & {\color{Green}x_{2}} & {\color{Blue}x^{2}_{2}} & {\color{Cyan}x^{3}_{2}} & {\color{violet}x^{4}_{2}}\\
%%				{\color{Red}\vdots} & {\color{Green}\vdots} & {\color{Blue}\vdots}  & {\color{Cyan}\vdots} & {\color{violet}\vdots}   \\
%%				{\color{Red}1} & {\color{Green}x_{n}} & {\color{Blue}x^{2}_{n}} & {\color{Cyan}x^{3}_{n}} & {\color{violet}x^{4}_{n}}\\
%%			\end{array}\right]
%%			\quad \bm{\alpha} = \left[ \begin{array}{c}
%%				{\color{Red}\alpha_{0}} \\
%%				{\color{Green}\alpha_{1}} \\
%%				{\color{Blue}\alpha_{2}} \\
%%				{\color{Cyan}\alpha_{3}} \\
%%				{\color{violet}\alpha_{4}}
%%			\end{array}\right]
%%			\quad , \quad \bm{\mu} = \bm{X} \bm{\alpha}
%%		\end{eqnarray*}
%%	}
%%	
%%	\item<5-> We minimize: $|\bm{y} - \bm{X}\bm{\alpha}|^{2} \quad \Rightarrow \quad 
%%	\bm{X}'\bm{X}\bm{\alpha} = \bm{X}\bm{y} \quad \Rightarrow \quad 
%%	\hat{\bm{\alpha}} = (\bm{X}'\bm{X})^{-1} \bm{X}'\bm{y}$
%%\end{itemize}
%%\end{frame}
%%
%%
%%\begin{frame}\frametitle{Model matrices}
%%\vspace{-0.5cm}      
%%\only<1>{
%%\begin{figure}
%%	\begin{center}
%%		\includegraphics[scale=.47]{Figures/ModMat1.pdf}
%%	\end{center}
%%\end{figure}
%%} 
%%\only<2>{
%%\begin{figure}
%%	\begin{center}
%%		\includegraphics[scale=.47]{Figures/ModMat2.pdf}
%%	\end{center}
%%\end{figure}
%%}
%%\end{frame}
%%
%%
%%
%%\begin{frame}\frametitle{Can we do better?}
%%\begin{itemize}
%%\item<1->[]
%%\begin{itemize}
%%\item Simple basis is good for simple example
%%\medskip
%%\item Basis function (powers of $x$) are global
%%\medskip
%%\item Moving one end moves the other end too
%%\medskip
%%\item Unexpected wiggles
%%\medskip
%%\item The higher the degree the more is sensitive
%%\end{itemize}
%%\bigskip
%%\item<2->[]
%%\begin{itemize}
%%\item We seek for local basis
%%\medskip
%%\item Useful for more complex data
%%\medskip
%%\item No assumptions on the trend (let the data speak by themselves!)
%%\medskip
%%\item Smooth outcomes
%%\end{itemize}
%%\end{itemize}
%%\end{frame}
%%
%%\begin{frame}\frametitle{Slightly more complex example}
%%\vspace{-0.5cm}
%%\only<1>{
%%\begin{figure}
%%\begin{center}
%%\includegraphics[scale=.47]{Figures/SimData.pdf}
%%\end{center}
%%\end{figure}
%%}
%%\end{frame}
%%
%%
%%\begin{frame}\frametitle{Introducing $B$-splines}
%%\vspace{-0.2cm}
%%\begin{itemize}
%%\item<1-> Create a suitable basis $\Rightarrow$ (equidistant) $B$-splines:
%%\begin{footnotesize}
%%\begin{eqnarray*}
%%\bm{B} = \left[ \begin{array}{ccccc}
%%B_{1}(x_{1}) & B_{2}(x_{1}) & {\color<2->{Red}B_{3}(x_{1})}
%%& \ldots & B_{k}(x_{1})\\ 
%%B_{1}(x_{2}) & B_{2}(x_{2}) & {\color<2->{red}B_{3}(x_{2})}
%%& \ldots & B_{k}(x_{2})\\ 
%%\vdots & \vdots & {\color<2->{Red}\vdots} & \vdots & \vdots
%%\\ 
%%B_{1}(x_{n}) & B_{2}(x_{n}) & {\color<2->{Red}B_{3}(x_{n})}
%%& \ldots & B_{k}(x_{n})\\ 
%%\end{array}\right]
%%\end{eqnarray*}
%%\end{footnotesize}
%%%\vspace{-0.2cm}
%%\only<1>{
%%	\begin{figure}
%%		\begin{center}
%%\includegraphics[scale=.25]{Figures/SingleBsplEmpty.pdf}
%%		\end{center}
%%\end{figure}
%%}
%%\only<2>{\begin{figure}
%%\begin{center}
%%\includegraphics[scale=.25]{Figures/SingleBspl.pdf}
%%\end{center}
%%\end{figure}}
%%\only<3->{\begin{figure}
%%\begin{center}
%%\includegraphics[scale=.25]{Figures/BsplBases.pdf}
%%\end{center}
%%\end{figure}}
%%\item<4-> $E(\bm{y}) = \bm{\mu} = \bm{B}\,\bm{\alpha} \quad
%%\Rightarrow  \quad \hat{\bm{\alpha}} = (\bm{B}'\bm{B})^{-1}
%%\bm{B}'\bm{y}$ 
%%\end{itemize}
%%\end{frame}
%%
%%\begin{frame}\frametitle{Fitting with 20 $B$-splines}
%%\vspace{-0.5cm}
%%\only<1>{
%%\begin{center}
%%\includegraphics[scale=0.47]{Figures/DataBspl.pdf} 
%%\end{center}
%%}  
%%\only<2>{
%%\begin{center}
%%\includegraphics[scale=0.47]{Figures/DataBsplUnpen.pdf} 
%%\end{center}
%%}  
%%\end{frame}
%%
%%\begin{frame}\frametitle{Penalizing the coefficients: $P$-splines}
%%\begin{itemize}
%%\item<1-> Outcomes are not smooth, we could:
%%\begin{itemize}
%%\item take less $B$-splines
%%\item place each $B$-splines in specific positions
%%\item<2-> set a double goal:
%%\end{itemize}
%%\only<2->{
%%\begin{enumerate}
%%\item good fit to the data, i.e.~low least-squares: $S =
%%|\bm{y} - \bm{B}\bm{\alpha}|^{2}$ 
%%\item smooth curve, i.e.~low roughness: $R = |\bm{D}
%%\bm{\alpha}|^{2}$ 
%%\end{enumerate}
%%}
%%\item<3-> We balance this two object-functions:
%%\begin{eqnarray*}
%%S^{*} = S + \lambda R = |\bm{y} - \bm{B}\bm{\alpha}|^{2} +
%%\lambda |\bm{D} \bm{\alpha}|^{2}
%%\end{eqnarray*}
%%\item<4-> Given a $\lambda$, this is again a linear system of
%%equation with explicit solution: 
%%\begin{eqnarray*}
%%\hat{\bm{\alpha}} = (\bm{B}'\bm{B} + \lambda
%%\bm{D}'\bm{D})^{-1} \bm{B}'\bm{y}
%%\end{eqnarray*}
%%\end{itemize}
%%\end{frame}
%%
%%
%%\begin{frame}\frametitle{Including a penalty, $d=2$}
%%\vspace{-0.5cm}
%%\only<1>{
%%\begin{center}
%%\includegraphics[scale=0.47]{Figures/DataBspl.pdf} 
%%\end{center}
%%}  
%%\only<2>{
%%\begin{center}
%%\includegraphics[scale=0.47]{Figures/DataBsplUnpen.pdf} 
%%\end{center}
%%}  
%%\only<3>{
%%\begin{center}
%%\includegraphics[scale=0.47]{Figures/DataBsplPen.pdf} 
%%\end{center}
%%}  
%%\only<4>{
%%\begin{center}
%%\animategraphics[loop, autoplay, scale=0.47, palindrome]{5}{Figures/PenBsplAnim}{0}{12}
%%\end{center}
%%}
%%
%%\end{frame}
%%
%%
%%\begin{frame}\frametitle{$P$-splines for Poisson data}
%%\begin{itemize}
%%\item<1-> In demography we love counting $\Rightarrow$ $y_i \sim \mathcal{P}(\mu_{i})$ 
%%\medskip
%%\item<2-> We use a ``linear predictor'': $\bm{\eta} =\ln(\bm{\mu}) = \bm{B}\bm{\alpha}$ 
%%\medskip
%%\item<3-> We have a non-linear system of equations, in an iterative
%%process, at step $t+1$: 
%%\begin{eqnarray*}
%%\tilde{\bm{\alpha}}_{t+1} = (\bm{B}'\tilde{\bm{W}}_{t}\bm{B} + \lambda \bm{D}'\bm{D})^{-1}
%%\bm{B}'\hat{\bm{W}}_{t}\tilde{\bm{z}}_{t} \, ,
%%\end{eqnarray*}
%%where 
%%
%%$\tilde{\bm{z}} = (\bm{y} - \tilde{\bm{\mu}}) /
%%\tilde{\bm{\mu}} + \bm{B}\tilde{\bm{\alpha}}$ (working dependent variable)
%%
%%\medskip
%%
%%$\tilde{\bm{W}} = diag(\tilde{\bm{\mu}})$ (weight matrix)
%%\bigskip
%%\item<4-> We can include an offset (exposures for mortality data)
%%just changing: 
%%\begin{eqnarray*}
%%\bm{\mu} = \bm{e} \, \exp(\bm{\eta}) = \exp(\bm{B}\bm{\alpha} +
%%\ln(\bm{e}))
%%\end{eqnarray*}
%%\end{itemize}
%%\end{frame}
%%
%%
%%
%%\section{$P$-splines for forecasting}
%%
%%\begin{frame}[fragile]\frametitle{Data structure \& basic model}
%%% \vspace{-1cm}
%%\begin{columns}
%%\begin{column}{0.5\textwidth}
%%\vspace{-0.5cm}
%%\begin{center}
%%{\color{Blue}Observed data}\\\medskip
%%        \includegraphics[scale=.2]{Figures/SchemeDEA.pdf}
%%        \includegraphics[scale=.04]{Figures/SchemeWHITE.pdf}
%%        \includegraphics[scale=.2]{Figures/SchemeEXP.pdf}
%%\end{center}
%%\onslide<2->{
%%\centering {\color{Green}Assumption:}
%%$$
%%y_{ij} \sim \mathcal{P}(e_{ij} \,\mu_{ij})\, 
%%$$
%%}
%%\end{column}
%%% \vrule{}
%%\begin{column}{0.7\textwidth}
%%% \vspace{1cm}
%%\onslide<3->{
%%\centering{\color{Red}Model:}
%%}
%%\begin{itemize}
%%\setlength\itemsep{0em}
%%\item<3->[] $\bm{y} = \verb"vec"(\bm{Y})$
%%\item<3->[] $\bm{e} = \verb"vec"(\bm{E})$
%%\end{itemize}
%%\onslide<3->{
%%\vspace{-0.3cm}
%%\begin{eqnarray*}
%%\ln(\bm{y}) = \ln(\bm{e})+ \ln(\bm{\mu}) &=& \ln(\bm{e}) + \bm{\eta}\\
%%&=& \ln(\bm{e}) + \bm{B}\,\bm{\alpha}
%%\end{eqnarray*}
%%}
%%\vspace{-.7cm}
%%\begin{itemize}
%%\setlength\itemsep{0em}
%%\item<3->[] $\bm{B} = {\color{Blue}\bm{B}_{t_{1}}} \otimes {\color{Blue}\bm{B}_{a}}$
%%\item<3->[] $\bm{\alpha}$: penalized coefficients
%%\end{itemize}
%%\onslide<4->{
%%%\vspace{-.2cm}
%%\includegraphics[scale=.35]{Figures/KronBsplinesINK.pdf}
%%}
%%\end{column}
%%\end{columns}
%%\begin{small}
%%\vspace{-.1cm}
%%\begin{itemize}
%%\setlength\itemsep{0em}
%%\item<5-> estimated by a penalized iterative weighted least-squares algorithm
%%\end{itemize}
%%\end{small}
%%
%%\end{frame}
%%
%%
%%
%%\begin{frame}[fragile]\frametitle{Forecasting with $P$-splines}
%%\begin{itemize}
%%\item<1-> Forecasting is treated as a missing value problem
%%\item<2-> Data are augmented with arbitrary future values and we define a weight matrix
%%\item<3-> $B$-spline bases are augmented too: $\bm{B} = [{\color{Blue}\bm{B}_{t_{1}}}:{\color{Red}\bm{B}_{t_{2}}}] \otimes {\color{Blue}\bm{B}_{a}}$
%%\item<4-> The original algorithm can be adapted
%%\end{itemize}
%%\begin{itemize}
%%\pooritem<5-> No information about mortality structure is used
%%\end{itemize}
%%
%%\onslide<2->{
%%\begin{columns}
%%\begin{column}{0.55\textwidth}
%%\includegraphics[scale=.12]{Figures/SchemeDEAfor.pdf}
%%        % \includegraphics[scale=.02]{Figures/SchemeWHITE.pdf}
%%\includegraphics[scale=.12]{Figures/SchemeEXPfor.pdf}
%%        % \includegraphics[scale=.02]{Figures/SchemeWHITE.pdf}
%%\includegraphics[scale=.12]{Figures/SchemeWEIfor.pdf}
%%\end{column}
%%\begin{column}{0.6\textwidth}
%%\onslide<3->{\includegraphics[scale=.4]{Figures/KronBsplinesFORINK.pdf}}}
%%\end{column}
%%\end{columns}
%%
%%
%%\end{frame}
%%
%%% -----------------------------------------------------  %%
%%% -----------------------------------------------------  %%
%%\section{$CP$-splines}
%%
%%%% -----------------------------------------------------  %%
%%%% -----------------------------------------------------  %%
%%\begin{frame}\frametitle{Observing mortality patterns (over ages)}
%%\vspace{-0.3cm}
%%\begin{itemize}
%%\item<1-> Over ages we observe certain regularities in mortality shape
%%\vspace{0.3cm}
%%\only<1>{
%%	\begin{center}
%%\includegraphics[scale=.24]{Figures/UsaAgeShape.pdf}
%%\includegraphics[scale=.24]{Figures/UsaAgeDerivEmpty.pdf}
%%\end{center}
%%}
%%\only<2>{
%%	\begin{center}
%%\includegraphics[scale=.24]{Figures/UsaAgeShape.pdf}
%%\includegraphics[scale=.24]{Figures/UsaAgeDeriv.pdf}
%%\end{center}
%%}
%%\vspace{0.1cm}
%%\item<2-> Interested solely in shapes (regardless their levels), these
%%regularities are condensed in the (relative) derivatives
%%\end{itemize}
%%\end{frame}
%%
%%%% -----------------------------------------------------  %%
%%%% -----------------------------------------------------  %%
%%\begin{frame}[fragile]\frametitle{Observing mortality patterns (over years)}
%%\vspace{-0.3cm}
%%  \begin{itemize}
%%  \item<1-> Variations over time are larger than over age
%%  \item<2-> We can still observe regularities for each age
%%\only<1>{
%%		\begin{center}
%%        \includegraphics[scale=.18]{Figures/EmptyYear.pdf}
%%
%%        \includegraphics[scale=.18]{Figures/EmptyYear.pdf}
%%        	\end{center}
%%    }
%%\only<2>{
%%		\begin{center}
%%        \includegraphics[scale=.18]{Figures/UsaYearShape.pdf}
%%
%%        \includegraphics[scale=.18]{Figures/EmptyYear.pdf}
%%                	\end{center}
%%    }
%%\only<3>{
%%		\begin{center}
%%        \includegraphics[scale=.18]{Figures/UsaYearShape1.pdf}
%%
%%        \includegraphics[scale=.18]{Figures/UsaYearDeriv.pdf}
%%                	\end{center}
%%    }
%%\only<4>{
%%		\begin{center}
%%        \includegraphics[scale=.18]{Figures/UsaYearShape1.pdf}
%%
%%        \includegraphics[scale=.18]{Figures/UsaYearDerivCI.pdf}
%%                	\end{center}
%%    }
%%\only<5>{
%%		\begin{center}
%%\includegraphics[scale=.36]{Figures/UsaYearDerivCIallAges.pdf}       
%%        	\end{center}
%%    }
%%\item<3-> Interested in age-specific rate-of-change,
%%  i.e.~derivatives
%%\item<4-> Ignore small fluctuations $\Rightarrow$ we take 50\%
%%  confidence interval 
%%\end{itemize}
%%
%%\end{frame}
%%
%%
%% \begin{frame}[fragile]\frametitle{Working on the shape}
%% \begin{itemize}
%% \item<1-> Disregarding information about shapes seems unreasonable
%% \medskip
%% \item<2-> Future mortality {\color{Red}\emph{must}} follow data-driven
%%   age-profiles and rate-of-change
%% \medskip
%% \item<3-> We constrain derivatives of future mortality to lay
%%   within certain confidence intervals of observed derivatives
%% \medskip
%% \item<4-> Asymmetric penalties are employed for this purpose:
%% \begin{itemize}
%% % \item the main idea behind this penalty lays in its local behavior 
%% \item within each iteration, whenever current estimations present
%%   derivatives (in future years) out of the desired intervals, a
%%   penalty intervenes 
%% \end{itemize}
%% \medskip
%% \item<5> Let's see how it works in action for a specific year (2035) and a
%%   specific age (50)
%% \end{itemize}
%% \end{frame}
%%
%%\begin{frame}[fragile]\frametitle{Asymmetric penalty in action}
%%  \only<1>{
%%    \centering \textbf{Without} asymmetric penalty\\
%%%    \smallskip
%%%    \vspace{1.5cm}
%%%    \hspace{-0.5cm}
%%    \begin{center}
%%    \includegraphics[scale=.26]{Figures/NoAsyAder.pdf}
%%%    \includegraphics[scale=.01]{Figures/EmptyYear.pdf}
%%    \includegraphics[scale=.26]{Figures/NoAsyYderEmpty.pdf}
%%    \end{center}
%%  }
%%  \only<2>{
%%    \centering \textbf{Without} asymmetric penalty\\
%%%\smallskip
%%%    $\,$
%%%    \vspace{1.5cm}
%%%    \hspace{-0.5cm}
%%    \begin{center}
%%    \includegraphics[scale=.26]{Figures/NoAsyAder.pdf}
%% %   \includegraphics[scale=.01]{Figures/EmptyYear.pdf}
%%    \includegraphics[scale=.26]{Figures/NoAsyYder.pdf}
%%    \end{center}
%%  }
%%
%%  \only<3>{
%%	\centering \textbf{With} asymmetric penalty\\
%%	%\smallskip
%%	%    $\,$
%%	%    \vspace{1.5cm}
%%	%    \hspace{-0.5cm}
%%	\begin{center}
%%		\includegraphics[scale=.26]{Figures/AsyAder.pdf}
%%		%   \includegraphics[scale=.01]{Figures/EmptyYear.pdf}
%%		\includegraphics[scale=.26]{Figures/AsyYder.pdf}
%%	\end{center}
%%}
%%  \only<4>{
%%	\centering \textbf{With} asymmetric penalty\\
%%\smallskip
%%    \begin{center}
%%      \animategraphics[autoplay, scale=0.25, loop]{1}{./Figures/AnimLoop}{0}{14}
%%    \end{center}
%%  }
%%
%%\end{frame}
%%
%%
%%\section{Results}
%%
%%\begin{frame}[fragile]\frametitle{Outcomes for US, males}
%%
%%    \only<1>{
%%\vspace{-0.5cm}
%%      \begin{center}
%%        \includegraphics[scale=.32]{Figures/OutcomesUsaYEAR.pdf}
%%\begin{footnotesize}
%%Log-mortality. Ages 0-105, observed years 1960-2016, forecast up to 2050 
%%\end{footnotesize}
%%      \end{center}
%%    }
%%    \only<2>{
%%\vspace{-0.5cm}
%%      \begin{center}
%%        \includegraphics[scale=.32]{Figures/OutcomesUsaAGE.pdf}
%%\begin{footnotesize}
%%Log-mortality. Ages 0-105, observed years 1960-2016, forecast up to 2050 
%%\end{footnotesize}
%%      \end{center}
%%
%%    }
%%  \only<3>{
%%\vspace{-0.5cm}
%%      \begin{center}
%%        \includegraphics[scale=.26]{Figures/OutcomesUsaLE.pdf}
%%        \includegraphics[scale=.26]{Figures/UsaAgeDerivEmpty.pdf}
%%\smallskip
%%
%%\begin{footnotesize}
%%Life Expectancy. \\Ages 0-105, observed years 1960-2016, forecast up to 2050 
%%\end{footnotesize}
%%      \end{center}
%%  }
%%  \only<4>{
%%\vspace{-0.5cm}
%%      \begin{center}
%%        \includegraphics[scale=.26]{Figures/OutcomesUsaLE.pdf}
%%        \includegraphics[scale=.26]{Figures/OutcomesUsaEDINK.pdf}
%%\smallskip
%%
%%\begin{footnotesize}
%%Life Expectancy and Lifespan variability measure
%%($e_{0}^{\dagger}$). \\Ages 0-105, observed years 1960-2016, forecast
%%up to 2050  
%%\end{footnotesize}
%%      \end{center}
%%  }
%%
%%\end{frame}
%%
%%
%%\begin{frame}[fragile]\frametitle{Outcomes for Denmark, females}
%%
%%    \only<1>{
%%\vspace{-0.5cm}
%%      \begin{center}
%%        \includegraphics[scale=.32]{Figures/OutcomesDnkYEAR.pdf}
%%\begin{footnotesize}
%%Log-mortality. Ages 0-105, observed years 1960-2016, forecast up to 2050 
%%\end{footnotesize}
%%      \end{center}
%%    }
%%    \only<2>{
%%\vspace{-0.5cm}
%%      \begin{center}
%%        \includegraphics[scale=.32]{Figures/OutcomesDnkAGE.pdf}
%%\begin{footnotesize}
%%Log-mortality. Ages 0-105, observed years 1960-2016, forecast up to 2050 
%%\end{footnotesize}
%%      \end{center}
%%
%%    }
%%  \only<3>{
%%\vspace{-0.5cm}
%%      \begin{center}
%%        \includegraphics[scale=.26]{Figures/OutcomesDnkLE.pdf}
%%        \includegraphics[scale=.26]{Figures/UsaAgeDerivEmpty.pdf}
%%\smallskip
%%
%%\begin{footnotesize}
%%Life Expectancy. \\Ages 0-105, observed years 1960-2016, forecast up to 2050 
%%\end{footnotesize}
%%      \end{center}
%%  }
%%  \only<4>{
%%\vspace{-0.5cm}
%%      \begin{center}
%%        \includegraphics[scale=.26]{Figures/OutcomesDnkLE.pdf}
%%        \includegraphics[scale=.26]{Figures/OutcomesDnkEDINK.pdf}
%%\smallskip
%%
%%\begin{footnotesize}
%%Life Expectancy and Lifespan variability measure
%%($e_{0}^{\dagger}$). \\Ages 0-105, observed years 1960-2016, forecast
%%up to 2050  
%%\end{footnotesize}
%%      \end{center}
%%  }
%%
%%\end{frame}
%%
%%\section{Conclusions}
%%
%%
%%\begin{frame}[fragile]\frametitle{What I haven't shown here, $\ldots$}
%%\begin{itemize}
%%	\item[]<1->
%%	\begin{itemize}
%%		\item All associated equations
%%		\medskip
%%		\item How we address infant mortality in a smoothing setting
%%		\medskip
%%		\item Bootstrap procedure to obtain confidence interval
%%		\medskip		
%%		\item Out-of-sample performance
%%		\medskip
%%		\item Comparison with other alternative methods
%%		\medskip
%%		\item Effect of changing time-window on the outcomes
%%		\medskip
%%		\item Sensitivity analysis on confidence level in rate-of-change over time 
%%		\medskip
%%		\item Reproducible \texttt{R}-code
%%	\end{itemize}
%%\medskip
%%\item<2-> $\ldots$ but you can find in the paper below:\\
%%\smallskip
%%Camarda, C. G. (2019).\\
%%Smooth Constrained Mortality Forecasting. \\
%%\textit{Demographic Research}. \textbf{41} (38), 1091-1130\\
%%\end{itemize}
%%\end{frame}
%%
%%
%%
%%
%%
%%
%%
%%\begin{frame}\frametitle{Concluding remarks}
%%\begin{itemize}
%%\item<1-> We combine a powerful statistical methodology with prior
%%  demographic information
%%\medskip
%%\item<2-> From $P$-splines we gain good fit, flexibility and smooth
%%  outcomes
%%\medskip
%%\item<3-> With additional constraints, we incorporate knowledge about
%%  mortality shapes to guide future developments
%%  \medskip
%%\item<4-> We enforce shape constraints by asymmetric penalties on the
%%  observed mortality derivatives
%%  \onslide<5->{
%%  \begin{block}{}
%%  	\centering 	\color{Red}{\begin{Large}
%%  			$CP$-splines
%%  		\end{Large}}
%%  \end{block}	
%%}
%%\medskip
%%\item<6-> $CP$-splines can be adapted to achieve coherent mortality forecast for multiple sub-population
%%\end{itemize}
%%\end{frame}
%%
%%
%%
%%
%%
%%
%%
%%
%%
%%\section*{$\qquad$}
%%\section*{$\qquad$}
%%
%%\begin{frame}\frametitle{$\,$}
%%  \vspace{-1cm}
%%
%%  {\color{Red}
%%    \begin{center}
%%      \begin{tabular}{lll}
%%        $\qquad$$\qquad$$\qquad$$\qquad$ $\qquad$$\qquad$&
%%                                                           $\qquad$ \\  
%%        \hline
%%      \end{tabular}
%%    \end{center}
%%  }
%%
%%  \vspace{0.5cm}
%%  \begin{center}
%%    \begin{LARGE}
%%      Thanks for your attention.\\
%%      \smallskip
%%      Comments and questions?
%%    \end{LARGE}
%%  \end{center}
%%
%%  {\color{Red}
%%    \begin{center}
%%      \begin{tabular}{lll}
%%        $\qquad$$\qquad$$\qquad$$\qquad$ $\qquad$$\qquad$&
%%                                                           $\qquad$ \\  
%%        \hline
%%      \end{tabular}
%%    \end{center}
%%  }
%%\begin{center}
%%\begin{large}
%%\bigskip
%%
%%%\texttt{carlo-giovanni.camarda@ined.fr}
%%
%%
%%%\bigskip
%%
%%		{\color{Blue}\texttt{sites.google.com/site/carlogiovannicamarda}}
%%
%%
%%\end{large}
%%\end{center}
%%
%%
%%\end{frame}
%%



%\begin{frame}[fragile]\frametitle{Out-of-sample performance}
%\only<1>{
%\begin{itemize}
%\item Fit both Smooth LC and Constrained $P$-splines in 1960-2004
%\bigskip
%\item Forecast mortality 10 years ahead (2005-2014)
%\bigskip
%\item Compare forecast values to the observed ones
%\end{itemize}
%}
%
%    \only<2>{
%\vspace{-0.5cm}
%      \begin{center}
%        \includegraphics[scale=.37]{Figures/BackTestEtaUSAm.pdf}
%\smallskip
%
%\begin{footnotesize}
%Log-mortality. \\Ages 0-105, 1960-2004, forecast up to
%2014, observed years 1960-2014.  
%\end{footnotesize}
%      \end{center}
%    }
%\only<3>{
%\begin{itemize}
%\item Measure accuracy in $e_{0}$, $e_{0}^{\dagger}$ and log-rates
%  ($\bm{\eta}$) by 
%\smallskip
%\begin{itemize}
%\item mean absolute error (MAE): $\quad\;\,\,$ smaller $\Rightarrow$ better
%\item root mean square error (RMSE): $\,$smaller $\Rightarrow$ better
%\item mean error (ME): $\qquad\qquad\qquad$ closer to 0 $\Rightarrow$ better
%\end{itemize}
%\end{itemize}
%\vspace{-.4cm}
%\begin{scriptsize}
%\begin{center}
%\begin{tabular}{clcc|cc|cc}
%\multicolumn{1}{l}{}                &                                                 & \multicolumn{2}{c|}{MAE}                                                         & \multicolumn{2}{c|}{RMSE}                                                       & \multicolumn{2}{c}{ME}                                                         \\ \cline{3-8} 
%\multicolumn{1}{l}{}                &                                                 & DNK F                                & USA M                                & DNK F                                & USA M                                & DNK F                                & USA M                                \\ \hline\hline
%                                    & \cellcolor[HTML]{EFEFEF}Constrained $P$-S & \cellcolor[HTML]{EFEFEF}\textbf{0.882} & \cellcolor[HTML]{EFEFEF}\textbf{0.110} & \cellcolor[HTML]{EFEFEF}\textbf{0.967} & \cellcolor[HTML]{EFEFEF}\textbf{0.126} & \cellcolor[HTML]{EFEFEF}\textbf{0.882} & \cellcolor[HTML]{EFEFEF}\textbf{0.048} \\
%\multirow{-2}{*}{$e_{0}$}           & Smooth LC                               & 1.718                                  & 0.242                                  & 1.782                                  & 0.282                                  & 1.718                                  & 0.221                                  \\ \hline
%                                    & \cellcolor[HTML]{EFEFEF}Constrained $P$-S & \cellcolor[HTML]{EFEFEF}\textbf{0.298} & \cellcolor[HTML]{EFEFEF}\textbf{0.186} & \cellcolor[HTML]{EFEFEF}\textbf{0.317} & \cellcolor[HTML]{EFEFEF}\textbf{0.211} & \cellcolor[HTML]{EFEFEF}\textbf{-0.298} & \cellcolor[HTML]{EFEFEF}\textbf{0.186} \\
%\multirow{-2}{*}{$e^{\dagger}_{0}$} & Smooth LC                               & 0.925                                  & 0.412                                  & 0.942                                  & 0.426                                  & -0.925                                  & 0.412                                  \\ \hline
%                                    & \cellcolor[HTML]{EFEFEF}Constrained $P$-S & \cellcolor[HTML]{EFEFEF}\textbf{0.219} & \cellcolor[HTML]{EFEFEF}\textbf{0.055} & \cellcolor[HTML]{EFEFEF}\textbf{0.348} & \cellcolor[HTML]{EFEFEF}\textbf{0.078} & \cellcolor[HTML]{EFEFEF}\textbf{-0.162} & \cellcolor[HTML]{EFEFEF}\textbf{-0.010} \\
%\multirow{-2}{*}{$\bm{\eta}$}       & Smooth LC                               & 0.334                                  & 0.099                                  & 0.479                                  & 0.123                                  & -0.314                                  & -0.037                                 
%\end{tabular}
%\end{center}
%\end{scriptsize}
%}
%\end{frame}
%
%
%
%\begin{frame}[fragile]\frametitle{Changing time-window}
%\only<1>{
%\begin{itemize}
%\item Fit both Smooth LC and Constrained $P$-splines in $t_{0}$-2014
%\bigskip
%\item Forecast mortality up to 2050
%\bigskip
%\item Compare forecast values in 2050 for $e_{0}$ and
%  $e_{0}^{\dagger}$ by $t_{0}$
%\end{itemize}
%}
%
%    \only<2>{
%\vspace{-0.5cm}
%      \begin{center}
%        \includegraphics[scale=.37]{Figures/Window2.pdf}
%\smallskip
%
%\begin{footnotesize}
%    Life expectancy at birth (left) and measure of lifespan
%    variability ($e^{\dagger}_{0}$, right) in 2050. Modelled period
%    $t_{0}$ to 2014. Year $t_{0}$ 
%    given on horizontal axis. 
%\\    USA males and Denmark females, ages
%    0-105.
%\end{footnotesize}
%
%      \end{center}
%    }
%
%
%\end{frame}



%\begin{frame}[fragile]\frametitle{Confidence level in rate-of-change
%    over time ($\bm{\delta}^{t_{1}}$)}
%    \only<1>{
%\vspace{-0.5cm}
%      \begin{center}
%        \includegraphics[scale=.37]{Figures/Perce0ed.pdf}
%\smallskip
%
%\begin{footnotesize}
%    Life expectancy at birth (left) and measure of lifespan
%    variability ($e^{\dagger}_{0}$, right). \\Different level of
%    confidence for computing $\bm{\delta}_{L}^{t_{1}}$ and
%    $\bm{\delta}_{U}^{t_{1}}$. \\
%Danish females and US males, ages 0-105,
%years 1960-2014, forecast to 2050.
%\end{footnotesize}
%
%      \end{center}
%    }
%
%
%    \only<2>{
%\vspace{-0.5cm}
%      \begin{center}
%        \includegraphics[scale=.37]{Figures/PercAge.pdf}
%\smallskip
%
%\begin{footnotesize}
%    Death rates for age 35 over years. \\Different level of
%    confidence for computing $\bm{\delta}_{L}^{t_{1}}$ and
%    $\bm{\delta}_{U}^{t_{1}}$. \\
%Danish females and US males, ages 0-105,
%years 1960-2014, forecast to 2050.
%\end{footnotesize}
%
%      \end{center}
%    }
%
%
%\end{frame}



%
%%%%%%%%%%%%%%%%%%%%%%%%%%%%%%%%%%%%%%%%%%%%%%%%%%%%%%%%%%%%%%%%%%%%%%%%%%
%%%%%%%%%%%%%%%%%%%%%%%%%%%%%%%%%%%%%%%%%%%%%%%%%%%%%%%%%%%%%%%%%%%%%%%%%%
%
%%%%%%%%%%%%%%%%%%%%%%%%%%%%%%%%%%%%%%%%%%%%%%%%%%%%%%%%%%%%%%%%%%%%%%%%%
%\begin{frame}\frametitle{$\,$}
%\begin{center}
%\begin{LARGE}
%Additional slides
%\end{LARGE}
%\end{center}
%\bigskip
%\bigskip
%\bigskip
%\end{frame}
%
%
%\begin{frame}[fragile]\frametitle{The Penalized IWLS}
%  \begin{itemize}
%\item Given data:
%\begin{itemize}
%\setlength\itemsep{0em}
%\item[] $\bm{d} = \verb"vec"(\bm{D})$
%\item[] $\bm{e} = \verb"vec"(\bm{E})$
%\end{itemize}
%\item And model
%\vspace{-0.3cm}
%\begin{eqnarray*}
%\ln(\bm{d}) = \ln(\bm{e})+ \ln(\bm{\mu}) &=& \ln(\bm{e}) + \bm{\eta}\\
%&=& \ln(\bm{e}) + \bm{B}\,\bm{\alpha}
%\end{eqnarray*}
%\vspace{-.7cm}
%\begin{itemize}
%\setlength\itemsep{0em}
%\item[] $\bm{B} = \bm{B}_{y_{1}} \otimes \bm{B}_{x}$
%\item[] $\bm{\alpha}$: penalized coefficients
%\end{itemize}
%
%\item Estimate $\bm{\alpha}$ by penalized IWLS:
%$$
%(\bm{B}' \tilde{\bm{W}} \bm{B} + \bm{P}) \tilde{\bm{\alpha}} =
%\bm{B}'\tilde{\bm{W}}\tilde{\bm{z}}
%$$
%where 
%\begin{itemize}
%\item $\tilde{\bm{z}} = (\bm{d} - \bm{e}*\tilde{\bm{\mu}}) / \bm{e}*\tilde{\bm{\mu}}
%+ \tilde{\bm{\eta}}$
%\item $\tilde{\bm{W}} =\mathrm{diag}(\bm{e}*\tilde{\bm{\mu}})\,$
%\end{itemize}
%
%  \end{itemize}
%\end{frame}
%
%
%
%\begin{frame}[fragile]\frametitle{Asymmetric penalty in formulas over ages/1}
%\begin{footnotesize}
%\begin{itemize}
%
%\item $$
%\frac{\frac{\partial}{\partial \bm{a}} \hat{\bm{\mu}}}{\hat{\bm{\mu}}} =
%\frac{\partial}{\partial \bm{a}} \ln(\hat{\bm{\mu}}) =
%\frac{\partial}{\partial \bm{a}} \hat{\bm{\eta}} =
%\bm{D}^{t_{1}}_{a}\,\hat{\bm{\alpha}} \, ,
%$$
%where
%$$
%\bm{D}^{t_{1}}_{a} = \bm{B}_{t_{1}} \otimes \bm{C}_{a} \quad
%\mbox{and} \quad \bm{C}_{a} = \frac{1}{h} \left[ \,^{q-1}\bm{B}^{k}_{a} - \,^{q-1}\bm{B}^{k-1}_{a} \right]
%$$
%with $h$, $q$ and $k$ being knot-distance, degree and positions of
%the original $B$-spline basis, $\bm{B}_{a}$. 
%\smallskip
%
%\item $\bm{\delta}^{a}_{L}$ and $\bm{\delta}^{a}_{U}$: lower and upper
%  bounds of CI of the derivatives
%
%\item Keep same constraints for all years $\Rightarrow$ augment $\bm{\delta}$ over both dimensions:
%$$
%\begin{array}{ccc}
%\bm{g}^{a}_{L} &=& \bm{1}_{n_{1}+n_{2}} \otimes
%\bm{\delta}^{a}_{L}\\ 
%\bm{g}^{a}_{U} &=& \bm{1}_{n_{1}+n_{2}} \otimes
%\bm{\delta}^{a}_{U}
%\end{array}
%$$
%
%\item Similar computation is performed over years
%\end{itemize}
%\end{footnotesize}
%
%\end{frame}
%
%
%
%
%\begin{frame}[fragile]\frametitle{Asymmetric penalty in formulas over ages/2}
%\begin{footnotesize}
%\begin{itemize}
%\item New penalized IWLS:
%$$
%(\breve{\bm{B}}' \bm{V} \tilde{\bm{W}} \breve{\bm{B}} + \bm{P} +
%\bm{P}^{a} + \bm{P}^{t}) \tilde{\bm{\alpha}} =
%\breve{\bm{B}}'\bm{V} \tilde{\bm{W}}\tilde{\bm{z}} + \bm{p}^{a} +
%\bm{p}^{t}\, .
%$$
%where
%$$
%\begin{array}{ccc}
%\bm{P}^{a} &=& \bm{P}^{a}_{L} + \bm{P}^{a}_{U}\\
%\bm{P}^{t} &=& \bm{P}^{t}_{L} + \bm{P}^{t}_{U}
%\end{array}
%\qquad \mbox{and} \qquad 
%\begin{array}{ccc}
%\bm{p}^{a} &=& \bm{p}^{a}_{L} + \bm{p}^{a}_{U}\\
%\bm{p}^{t} &=& \bm{p}^{t}_{L} + \bm{p}^{t}_{U}
%\end{array}\, .
%$$
%
%\item As example, lower bounds over ages: 
%$$
%\begin{array}{ccl}
%\bm{P}^{a}_{L} &=& \kappa \;
%                   \bm{D}_{a}^{t_{1}+t_{2}}'\;\verb|diag|(\bm{s}\, \bm{v}^{a}_{L})\;\bm{D}^{t_{1}+t_{2}}_{a} \\
%  \bm{p}^{a}_{L} &=& \kappa \;
%                     \bm{D}_{a}^{t_{1}+t_{2}}'\;\verb|diag|(\bm{s}\,
%                     \bm{v}^{a}_{L})\; \bm{g}^{a}_{L}
%\end{array}
%\quad \mbox{with} \quad 
%\bm{v}^{a}_{L} = \begin{cases} 0 & \mbox{if}\quad \bm{D}^{t_{1}+t_{2}}_{a}\tilde{\bm{\alpha}} \geqslant \bm{g}^{a}_{L}\\
%                     1 & \mbox{if}\quad \bm{D}^{t_{1}+t_{2}}_{a}\tilde{\bm{\alpha}} < \bm{g}^{a}_{L}
%\end{cases}
%$$
%where
%$$
%\bm{v}^{a}_{L} = \begin{cases} 0 & \mbox{if}\quad \bm{D}^{t_{1}+t_{2}}_{a}\tilde{\bm{\alpha}} \geqslant \bm{g}^{a}_{L}\\
%                     1 & \mbox{if}\quad \bm{D}^{t_{1}+t_{2}}_{a}\tilde{\bm{\alpha}} < \bm{g}^{a}_{L}
%\end{cases}  \, ,
%$$
%and $\bm{s}$ is a 0/1 vector equal to 1 when the constraint is to be
%applied (future years). 
%\end{itemize}
%\end{footnotesize}
%
%\end{frame}
%
%
